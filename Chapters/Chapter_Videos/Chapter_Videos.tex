%************************************************
\chapter{Videos}\label{ch:videos}
%************************************************
Three videos are included with the thesis in the \emph{Videos.zip} archive. They are also available at \url{http://karolyszanto.ro/MastersThesis/videos/}. There is one video for each evaluation scenario, sequentially following the steps each participant had to complete. In each video, on the right side we have recorded the running simulator while in the left side we have recorded the ContextClient reflecting the current SSM context model of the system.\\

You can closely follow how entities get classified as the agent is interacts with the environment. Moreover, you can notice what the agent can and cannot do according to the current context: cannot interact with an item that is to far (the simulation gives a message), is an object is in the Action Space it can be interacted with, there is no default implementation for custom or combined interactions but you can notice that the system knows exactly what the agent is doing and displays proper messages.\\

\paragraph{WarmUpTask.mov} is a recording of running the simulation in the first evaluation prototype as described in Appendix \ref{sec:eval_warmup_scenario}. On the right side you can see the running simulation where the agent is interacting with the environment, while on the left side you can see the changes in the SSM context model as reflected in the ContextClient.

\paragraph{ALFTask.mov} is a recording of running the simulation in the second evaluation prototype as described in \ref{sec:eval_alf_scenario}. On the right side you can see the running simulation where the agent is interacting with the environment, while on the left side you can see the changes in the SSM context model as reflected in the ContextClient.

\paragraph{ChildproofTask.mov} is a recording of following the steps described in the third evaluation task, in order to create a new simulation using EgoSim, as described in Appendix \ref{sec:eval_childproof_scenario}. We have recorded the actual steps of implementing a simulation using the EgoSim framework. Besides following the steps in the description, at the end of the video we have demonstrated how you can extend the embedded support for combined interaction. We have simply displayed a message using EgoSim's notification manager.