%************************************************
\section{Scenario} % (fold)
\label{sec:scenario}
%************************************************
To better visualize the importance of the EgoSim framework, in this section we present a concrete scenario which has been used during the framework's evaluation. The labels in the scenario's description will be referenced in the requirements Section \ref{sec:requirements}.\\

The hypothetical problem presented in this scenario is that families cannot make their homes secure enough for their children. To provide a solution for this problem, there's a need for a system to constantly monitor the objects a child should keep away from and should not be interacting with. Based on this context information, we can further build various software service to secure the house from the child's actions. One example of such service is the "Secure Outlets Service" (SOS). This service has to detect whenever a child is approaching an outlet with a small object, in which case it should shut down the electricity switch for outlets, preventing the child from the possibility of getting electrocuted.\\

A good solution for this problem could be based on the egocentric interaction paradigm, as it can categorizes all the objects of interest around the human agent. Before implementing the system in a real world set-up, it is best to validate the design by simulating it. To visualize how the EgoSim framework could help, this is how we imagine a researcher would use it:

\begin{flushright}{\slshape
The researcher starts out by setting up the simulation, using the EgoSim framework. First, she/he identifies the target environment -- it resembles a living room where the child spends most of the time. Next, she/he 
needs a virtual model of the environment \textlabel{(1)}{scenario:1}. The model is populated \textlabel{(1A)}{scenario:1A} with everyday physical objects \textlabel{(1B)}{scenario:1B} and devices \textlabel{(1C)}{scenario:1C}. Some of the simulated entities have to be constantly monitored \textlabel{(1D)}{scenario:1D} by the system. Moreover, some of the entities should have the option to be picked-up \textlabel{(1E)}{scenario:1E} and put-down \textlabel{(1F)}{scenario:1F} onto surfaces \textlabel{(1G)}{scenario:1G} (e.g. a table, the floor, etc). Optionally, picked-up entities could interact with other entities \textlabel{(1H)}{scenario:1H}, other than surfaces. Finally, the monitored entities need to provide the necessary configuration information required by SSM model in the egocentric interaction paradigm \textlabel{(1I)}{scenario:1I}.
} \\ \medskip
\end{flushright}

\begin{flushright}{\slshape
After the researcher has finished setting up the simulation, using the EgoSim framework, she/he runs the simulation in order to test out the system design. Once the simulation is started up, the researcher finds himself impersonating a child, the entity this system is designed for. The child is represented by a virtual avatar \textlabel{(2)}{scenario:2}, placed within the simulated target environment, in this case the living room. The avatar can be moved around the living room \textlabel{(2A)}{scenario:2A} and can look around to inspect the surroundings \textlabel{(2B)}{scenario:2B}. If the agent is close enough to an entity, it can be interacted with (i.e. picked-up, put-down, etc) \textlabel{(2C)}{scenario:2C}. As any of the previous actions are carried out, the SSM monitoring service \textlabel{(2D)}{scenario:2D} keeps track of the agent's position as well as the distance to each object in the agent's visual spectrum (what can be seen at a certain moment in time). Based on this information, the entities are being categorised, in real time, into SSM sets \textlabel{(2E)}{scenario:2E}.
} \\ \medskip
\end{flushright}

\begin{flushright}{\slshape
The Secure Outlets Service is implemented as a third party service, listening to changes in the SSM sets through the EgoSim's API provided by the running simulation. As the simulation unfolds, the researcher controls the avatar to pick up a pen located on the living room table. With the pen picked up, the agent is controlled to approach one of the simulated outlets; if close enough, the service detects an immediate possible threat, as there is a high possibility the child could insert the pen into the outlet. The service shuts down the electricity switch for outlets, preventing the child from the possibility of getting electrocuted \textlabel{(3)}{scenario:3}.
} \\ \medskip
\end{flushright}

The points in the above scenario marked with (1-3) will help motivating the high level requirements we will come up with for EgoSim in Section \ref{sec:requirements}.
% subsection scenario (end)