%************************************************
\section{Goal} % (fold)
\label{sec:goal}
%************************************************
The goal of this project is to design and implement an open-source simulation framework that can help developers of mobile context-aware systems to explore design alternatives. The framework has to dynamically classify physical objects, virtual objects and mediators that are close to the human agent according to the situative space model (SSM) \cite{pederson2011situative}, allowing the context-aware system developer to design system logic on the basis of this classification. The human agent is to be represented as an avatar that can move around and perform basic interaction with objects in the simulated environment.\\

The requirements of the framework are:
\begin{enumerate}
	
	\item[\textlabel{1.}{goal:1}] Allow researchers to easily build a simulated environment of their target mobile context-aware systems, based on situative space model concepts. The researchers will simulate the interaction of the end users with the egocentric system, controlling a virtual avatar in the simulated environment to move around and interact with object.

	\item[\textlabel{2.}{goal:2}] Classify dynamically, in real-time, the monitored objects that are close to the human agent according to the situative space model, keeping the SSM spaces updated. The classification algorithms will be triggered as the agent interacts with the environment (move around, look around, pick up objects, etc).

	\item[\textlabel{3.}{goal:3}] Provide an API for third party services to query the current state of the SSM spaces. The API will empower third party client services to query the status of a particular space or all spaces at once. Moreover, on top of the API, the framework will offer an easy way to visualize the SSM spaces in real-time.

	\item[\textlabel{4.}{goal:4}] To be open-source and freely available.
\end{enumerate}

% The objects simulated by the framework will have the following characteristics:
% \begin{itemize}
% 	\item they can be physical objects, virtual object (i.e. software) and mediators (i.e. devices, sensors)
% 	\item they can be composed; for example a lamp can be placed on a table, an electronic calendar can be placed on the wall
% 	\item their state is made up by attributes like: position (within the simulated environment), weight, relations with other objects (i.e. pen placed on top of a paper), and various other specific state attributes (e.g. on/off for a lamp)
% \end{itemize}

% For the agent, the framework will monitor a set of characteristics and allow a certain set of actions to be carried out. The following statements present details:
% \begin{itemize}
% 	\item the framework will monitor location within the simulated environment, visual spectrum (the objects within it), distance to each perceived object, body position (sitting/standing)
% 	\item the agent's movement can be easily controlled in an intuitive user interface
% 	\item the agent can pick up objects if possible (based on weight, proximity etc.), carry objects around, put objects down
% \end{itemize}

Chapter \ref{ch:design} details the solutions that we propose to accomplish these goals.
% subsection goal (end)