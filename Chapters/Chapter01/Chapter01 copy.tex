%************************************************
\chapter{Introduction}\label{ch:introduction}
%************************************************

Before the early 1970s, computers were owened by large institutions like corporations, universities or government agencies. These machines, alos know as \emph{Mainframes}, were costly and space consuming, most institutions owning only one piece of such equipment. Lacking any kind of user interface, Mainframes would accept jobs in the form of punched cards, an early input mechanism for data and programs into computers. A group of people in the institutions were instructed how to use the machines (creating punched cards, operating the machines) and they represented the interface between everyday users and the Mainframe. Quite a bottleneck!\\

In 1971 the first commertial microprocessor was released, representing the rise of a new era: the Personal Computer (PC). The PC represented a huge step forward, offering in perspective the possibility for anyone to own a computer and to operate it using novel input/output mechanism: graphical user interface, mouse, keyboard and other peripherals. This is the standard PC setup that we know and, after such o along time, still use. By this I want to point that although we have came a long way, by hardware and software advancements, the human computer interaction (HCI) form that is mostly used to interact with PCs is just as it was envisioned a little more than 40 years ago.\\

The PCs revolution was followed by the idea of portable personal devices like laptops, dynabooks (an educatoinal device simmilar to a nowadays tablet). Although novel at that time, all the fever generated by this revolution was viewed by a group of people at PARC\footnote{Palo Alto Research Center} as being ephemeral: \"[...] My colleagues and I at PARC think that the idea of a \"personal computer\" itself is misplaced, and that the vision of laptop machines, dynabooks and \"knowledge navigators\" is only a transitional step toward achieving the real potential of information technology. Such machines cannot truly make computing an integral, invisible part of the way people live their lives. Therefore we are trying to conceive a new way of thinking about computers in the world, one that takes into account the natural human environment and allows the computers themselves to vanish into the background\". \cite{weiser1991computer}. This futuristic concept envisioned at PARC which further shaped the way we see and use technology and devices is known as \emph{ubicomp}\footnote{Ubiquitous Computing}.\\

\section{Problem Statement} % (fold)
\label{sec:problem_statement}

Designing and developing mobile context-aware systems in a real-life set-up represents a tedious, costly and time consuming process. In addition, most systems require testing and experimentation in their target location for valid and beneficial results to be produced; however it is difficult and costly to set up a certain location to support the designed system. Moreover, incorporating physical objects into the system design, adds up to the aforementioned challenges.

% These facts impose a challenge for researchers to efficiently work with the Egocentric Interaction Paradigm. 

% To overcome these challenges, I am designing and developing a simulation environment for mobile context-aware system design, focusing on the egocentric paradigm. The framework will monitor the agent's context in regards to the world space and the context of surrounding entities.\\

% subsection problem_statement (end)

%************************************************
\section{Goal} % (fold)
\label{sec:goal}
%************************************************
The goal of this project is to design and implement the basis of an open-source simulation environment for mobile context-aware design using Java based technologies. The framework will capture the context of physical objects, virtual objects and mediators \cite{pederson2011situative} in regards to the agent's parameters. The agent represents an avatar of the user interacting with the simulated environment.\\

The following set of goals will guide our work:
\begin{enumerate}
	\item Host all the work in a publicly available source control repository
	\item Follow a modular design to implement the core of the framework as a set of decoupled components:
		\begin{itemize}
			\item The \emph{Data Model}. This represents the contextual entities and the relationship between them that the simulator will monitor, briefly detailed in Section \ref{sec:data_model}.
			\item The \emph{World Model}. Represents the current status of the simulated environment (e.g. position of the agent, status of a certain switch etc.). The structure of the entities in this model have been defined in the Data Model.
			\item A universal communication protocol to access the core. It needs to be universal so that other components, can easily be developed and integrated.
			\item Query and modify API of information in the World Model. To make the integration between the core of the framework and the other modules, these APIs will need to provide a complete set of methods.
		\end{itemize}
	\item Evaluate the framework:
		\begin{itemize}
			\item Implement a module to modify the state of the simulated environment (DummySim). This comprises of manipulating the agent's parameters and performing actions like picking up an item, touching a screen etc. The purpose of this module is to evaluate the modify API of the core. It also servers as a tool to modify the environment which is a required input for the SSM model to be evaluated.
			\item Implement the \emph{SSM Model}. Using the current World Model's query API, retrieve the current set of contextual data and apply the rules defined in the SSM. This component will actively monitor the current status of the simulator, having as a result an up to date collection of SSM entity sets: world space, perception space, recognizable set, examinable set, action space, selected set and manipulated set. This evaluation will be considered successful if the results are similar to the results obtained in the initial evaluation of the SSM described in \cite{pederson2011situative}.
		\end{itemize}
\end{enumerate}

The followings are not target of this work:
\begin{itemize}
	\item Create a close-to-real world interaction. Projects that simulate real world scenarios, employ game alike solution providing complex graphics and interaction between the user of the simulator and the simulated world. This can be viewed as a future work; in the current work I am focusing on the framework's core.
\end{itemize}

\subsection{Data Model}\label{sec:data_model}
The data model is aimed at defining the supported entities and the relationships between them. In this work the aim is to support physical objects, virtual objects and mediators.\\

The agent represents a virtual character that can freely moved around in the environment. Beside the position the of the agent, the user can also control more complex parameters of the agent like orientation (affects field of vision), sitting down / standing, hands orientation etc. The agent will be also be able to carry out a certain set of actions like picking objects up, putting objects down etc.\\

Physical objects will be described as entities with a certain set of properties (location, weight etc.). Entity composition will be supported to some extend: place objects on top of each other (i.e. a stapler on a table), a mobile phone in the backpack worn by the agent.\\

This section created a brief overview of the supported entity types and relationships. They will be described in details in Chapter \ref{ch:design}.
% subsection goal (end)

%************************************************
\section{Method} % (fold)
\label{sec:method}
%************************************************

In order to achieve the goals detailed in Section \ref{sec:goal}, I am going to follow an iterative design process \cite{mackay1997hci}. Each iteration will comprise of two steps:
\begin{enumerate}
	\item prototyping
	\item evaluation
\end{enumerate}

As part of the process, there will be three iterations:
\begin{itemize}
	\item Early prototype. This step represents the skeleton of the system, where I implemented a first version of the simulation framework; \emph{EgoSim}. As part of the implementation, I have set up a trivial test environment which provides input to the underlying framework to run the SSM classifications. The resulting SSM sets are visualized, in real time, in the \emph{ContextClient}. This prototype has been evaluated in close collaboration with the supervisors of this thesis.
	\item High fidelity prototype. Based on the input from the evaluation of the early prototype, I modified the system accordingly. Further, I have implemented all the intended interaction mechanism between the agent and the environment and I have also developed a public API to provide easy access to the SSM spaces for third party services. As part of this prototype, I have designed an advanced simulation environment based on a real-life inspired scenario. This prototype has been evaluated by the target group of 17 participants. The purpose of the evaluation was to test the usability of the simulator, the responsiveness of the ContextClient, the correctness and usability of the SSM sets, as well as how intuitive is the interaction with the environment.
	\item Final product. In this final iteration I have fixed the bugs discovered during the previous iteration. Furthermore, I have designed a comprehensive documentation on how to use the EgoSim framework. For this iteration I have created a third simulation environment and, this time, the 17 participants have evaluated the usefulness and ease-of-use of the EgoSim framework, by setting up a simulation based on the framework and the documentation. As most participants were not experts of the Egocentric Interaction Paradigm, I have also evaluated framework as an educational tool by comparing the participants understanding of SSM related concepts before and after the evaluation.
\end{itemize}

I will wrap up our research by drawing conclusion based on our findings and establishing goals for future work.

% subsection method (end)

\section{Thesis Overview} % (fold)
\label{sub:thesis_overview}

The rest of the work is structured as follows: Chapter \ref{ch:related_work} reviews similar systems, with respect to the criteria defined in the current work.\\

Chapter \ref{ch:design} and \ref{ch:implementation} constitute the core of the thesis. Chapter \ref{ch:design} presents the design of the system, while Chapter \ref{ch:implementation} discusses the implementation details of the practical work.\\

Chapter \ref{ch:evaluation} is dedicated to the evaluation of EgoSim. Chapter \ref{ch:conclusion} summarizes the contributions brought by the this thesis and presents the main goals of our future work.

% subsection thesis_overview (end)

