%************************************************
\chapter{Introduction}\label{ch:introduction}
%************************************************

Before the early 1970s, computers were owened by large institutions like corporations, universities or government agencies. These machines, alos know as \emph{Mainframes}, were costly and space consuming, most institutions owning only one piece of such equipment. Lacking any kind of user interface, Mainframes would accept jobs in the form of punched cards, an early input mechanism for data and programs into computers. A group of people in the institutions were instructed how to use the machines (creating punched cards, operating the machines) and they represented the interface between everyday users and the Mainframe. Quite a bottleneck!\\

In 1971 the first commertial microprocessor was released, representing the rise of a new era: the Personal Computer (PC). The PC represented a huge step forward, offering in perspective the possibility for anyone to own a computer and to operate it using novel input/output mechanism: graphical user interface, mouse, keyboard and other peripherals. This is the standard PC setup that we know and, after such o along time, still use. By this I want to point that although we have came a long way, by hardware and software advancements, the human computer interaction (HCI) form that is mostly used to interact with PCs is just as it was envisioned a little more than 40 years ago.\\

The PCs revolution was followed by the idea of portable personal devices like laptops, dynabooks (an educatoinal device simmilar to a nowadays tablet). Although novel at that time, all the fever generated by this revolution was viewed by a group of people at PARC\footnote{Palo Alto Research Center} as being ephemeral: \"[...] My colleagues and I at PARC think that the idea of a \"personal computer\" itself is misplaced, and that the vision of laptop machines, dynabooks and \"knowledge navigators\" is only a transitional step toward achieving the real potential of information technology. Such machines cannot truly make computing an integral, invisible part of the way people live their lives. Therefore we are trying to conceive a new way of thinking about computers in the world, one that takes into account the natural human environment and allows the computers themselves to vanish into the background\". \cite{weiser1991computer}. This futuristic concept envisioned at PARC which further shaped the way we see and use technology and devices is known as \emph{ubicomp}\footnote{Ubiquitous Computing}.\\

\section{Problem Statement} % (fold)
\label{sec:problem_statement}

TODO: Will be a short description of the problem as highlighted in the Introduction.

% subsection problem_statement (end)

%************************************************
\section{Goal} % (fold)
\label{sec:goal}
%************************************************
The goal of this project is to design and implement an open-source simulation environment for mobile context-aware system design. The framework will monitor the context of physical objects, virtual objects and mediators \cite{pederson2011situative} in regards to the agent's parameters (i.e. location, field of vision, body position sitting / standing, objects within the agent's reach etc.). The agent represents an avatar of the user interacting with the simulated environment.\\

The followings represent the goal of my thesis project:
\begin{enumerate}
	\item The framework should be able to adequately represent aspects of the SSM and update the model in real time, based on the input of an event generator (i.e. the simulator).
	\item Implement a simulator to generate events which trigger contextual changes in the framework.
	\item Visualize, in real time, the SSM spaces, as contextual changes occur.
	\item Host all the work in a publicly available source control repository.
\end{enumerate}

The objects simulated by the framework will have the following characteristics:
\begin{itemize}
	\item they can be physical objects, virtual object (i.e. software) and mediators (i.e. devices, sensors)
	\item they can be composed; for example a lamp can be placed on a table, an electronic calendar can be placed on the wall
	\item more specialized devices can be represented, like wearable devices, hand-held devices (i.e. mobile phones, tablets)
	\item their state is made up by attributes like: position (within the simulated environment), height, weight, relations with other objects (i.e. pen placed on top of a paper), and various other specific state attributes (e.g. on/off for a lamp)
	\item some of the objects might receive data from external, real-life objects; for example, a simulated sensor motion sensor could have its data retrieved from a motion sensor installed in the same room with the researchers
\end{itemize}

For the agent, the framework will monitor a set of characteristics and allow a certain set of actions to be carried out. The following statements present details:
\begin{itemize}
	\item the framework will monitor location within the simulated environment, visual spectrum (the objects within it), distance to each perceived object, body position (sitting/standing)
	\item the agent's movement can be easily controlled in an intuitive user interface
	\item the agent can pick up objects if possible (based on weight, proximity etc.), carry objects around, put objects down
	\item objects can be in the agent's hand, pocket, backpack etc.
	\item the agent can interact with digital devices (e.g. gestures on a touch scree, press buttons)
\end{itemize}

At this time, it is very uncertain if I will be able to fulfill all the above goals and it is left up to the Design Chapter \ref{ch:design} to decide how we approach the goals of this research.
% subsection goal (end)

%************************************************
\section{Method} % (fold)
\label{sub:method}
%************************************************

% subsection method (end)

\section{Thesis Overview} % (fold)
\label{sub:thesis_overview}

The rest of the work is structured as follows: Chapter \ref{ch:related_work} reviews similar systems, with respect to the criteria defined in the current work.\\

Chapter \ref{ch:design} and \ref{ch:implementation} constitute the core of the thesis. Chapter \ref{ch:design} presents the design of the system, while Chapter \ref{ch:implementation} discusses the implementation details of the practical work.\\

Chapter \ref{ch:evaluation} is dedicated to the evaluation of EgoSim. Chapter \ref{ch:conclusion} summarizes the contributions brought by the this thesis and presents the main goals of our future work.

% subsection thesis_overview (end)

