%************************************************
\chapter{Introduction}\label{ch:introduction}
%************************************************

Before the early 1970s, computers were owned by large institutions like corporations, universities or government agencies. These machines, also know as \emph{Mainframes}, were costly and space consuming, most institutions owning only one piece of such equipment. Lacking any kind of intuitive user interface, Mainframes would accept jobs in the form of punched cards, an early input mechanism for data and programs into computers. Only a group of people in each institution had the knowledge on how to operate the machines and they represented the interface between everyday users and the Mainframe. Quite a bottleneck!\\

In 1971, the first commercial microprocessor was released, representing the rise of a new era: the Personal Computer (PC). The PC represented a huge step forward, offering in perspective the possibility for anyone to own a computer and to operate it using a novel input/output mechanism: graphical user interface, mouse, keyboard and other peripherals. This is the standard PC setup that we know and, after such a long time, still use. By this we want to point out that although we have came a long way, by hardware and software advancements, the human computer interaction (HCI) that is mostly used to interact with PCs is the same as envisioned a little more than 40 years ago.\\

The PCs revolution was followed by the idea of portable personal devices like laptops and dynabooks (an educational device similar to a nowadays tablet). Although novel at that time, all the fever generated by this revolution was viewed by a group of people at PARC\footnote{Palo Alto Research Center} as being ephemeral: ''[...] My colleagues and I at PARC think that the idea of a ''personal computer'' itself is misplaced, and that the vision of laptop machines, dynabooks and ''knowledge navigators'' is only a transitional step toward achieving the real potential of information technology. Such machines cannot truly make computing an integral, invisible part of the way people live their lives. Therefore we are trying to conceive a new way of thinking about computers in the world, one that takes into account the natural human environment and allows the computers themselves to vanish into the background''. \cite{weiser1991computer}. This futuristic concept envisioned at PARC, which further shaped the way we see and use information technology and devices, is known as Ubiquitous Computing, or''Ubicomp'' for short.\\

The novel vision in Ubicomp can be viewed as a second major advancement in the world of information technology. The first major advancement, the transition from Mainframe to PC, took the relation between people and computer from a many-to-one to a one-to-one relationship. Ubicomp envisioned a transition from one-to-one to one-to-many, where each user will own not one but many device. These device would range from PCs to all sorts of portable device, some specialized at accomplishing specific tasks and performing under various circumstances. Classical HCI was not fit for this new set-up. More advanced and more intuitive concepts were needed to make the best of these technological advances. One such emerging concept was \emph{context-aware computing}. Context is ''any information that can be used to characterize the situation of an entity, where an entity can be a person, place, or physical or computational object. We define context-awareness or context-aware computing as the use of context to provide task-relevant information and/or services to a user. Three important context-awareness behaviours are the presentation of information and services to a user, automatic execution of a service, and tagging of context to information for later retrieval''. \cite{abowd1999towards}. Context-awareness takes HCI to a whole different level. We are not thinking anymore of only sitting down in front of a computer and interacting with it using a few peripherals. Users and developers have a whole new spectrum they can explore, in a dynamic and continuously evolving technological environment. Take for example a piece of software that gives information on historical monuments. If this software is running on a PC, the user would manually search for a specific monument to retrieve the information. The same software running on a mobile device with localization capabilities (i.e. GPS), could take away all the explicit interaction by determining the user's location and if this is in near proximity of a certain historical monument, automatically display the relevant information.\\

While designing mobile pervasive computing environments, software developers are bound to work with cutting edge hardware and software technologies and to adopt new communication models for HCI. Throughout the years, in Ubicomp, much of the effort has been devoted to address hardware and software challenges, hence, with the explosion of devices in the form of mobile phones and tablets having a wide range of sensing capabilities, modelling of context-aware software became \emph{device-centric}. This way, designers of mobile context-aware systems base their design mostly on device sensory data.\\

In our everyday lives we encounter and interact with our surrounding through a wide range of mobile devices and physical objects. While the current device-centric modelling makes it easy to incorporate devices, incorporating physical objects into a context-aware system design, remains a challenge. When trying to integrate physical objects into the system design, the designer is bound to think in terms of existing technologies that can facilitate their integration; for example, making a system aware of a chair, a designer might think about attaching various sensors to it (pressure, motion, proximity, etc). While the device-centric modelling makes it natural to think about devices, when it comes to integrate physical objects, it limits the designer instead of offering a natural way to solve the problem. To address the struggling on conceptually incorporating the real world into the system design, Pederson in \cite{pederson2010towards} introduced a \emph{body-centric} modelling framework that incorporates physical objects, mediators (i.e. devices) and virtual objects (part of a runnign software, like a window, a stream of sound, interacted with through mediators) of interest on the basis of proximity and human perception, framed in the context of an emerging ''egocentric'' interaction paradigm. ''Egocentric'' signals that it is the human body and mind of a specific human individual that acts as centre of reference to which all modelling is anchored in this interaction paradigm. Part of the egocentric paradigm is the Situative Space Model (SSM): an interaction model and a design tool which captures what a specific human agent can perceive and not perceive, reach and not reach, at any given moment in time.\\

The SSM is meant to categorize objects around the agent, into a number of sets, based on it's continuously evolving state as it interacts with the surrounding environment. The sets maintained by the SSM are, as described in \cite{pederson2010towards}:
\begin{itemize}
	\item World Space. Contains all physical objects, mediators and virtual objects to be part of a certain model.
	\item Perception Space (PS). Objects around the agent that can be perceived at a given moment (i.e. objects currently in the visual spectrum of the user).
	\item Recognizable Set (RS). Objects in the PS within recognition distance.
	\item Examinable Set (ES). Objects in the PS that are within their examination range.
	\item Action Space (AS). Objects surrounding the agent that are currently accessible to the agent's physical actions.
	\item Selected Set (SdS). Objects being currently physically or virtually handled (i.e. touched).
	\item Manipulated Set (MS). Objects whose state is currently being acted upon by the agent.
\end{itemize}

Evaluating the SSM was a tedious work and consisted in creating a real-life setup, with digitally enhanced physical objects (various sensors to track agent proximity and actions) while using a wide range of technologies to track the agent's current situation (body position, orientation, visual spectrum etc.). This is a costly and time consuming process making further development and evaluation of the egocentric paradigm a challenge. As Pederson also stated in \cite{pederson2011situative}: ''accurate tracking of objects and mediators needed for real-time application of the SSM will probably remain a challenge for years to come''.\\

To overcome these challenges, in this Master Thesis we aim at designing a simulation environment for mobile context-aware system design, focusing on the egocentric paradigm. The framework will monitor the agent's context in regards to the world space and the context of surrounding entities. Based on the design, we will implement the \emph{EgoSim} framework \cite{egosim:online}.\\ 

We are designing this system for Ubicomp researchers and developers designing mobile context-aware systems that change their behaviour on how the body of the agent is in regards to the surrounding environment and entities. To offer the most to this target community, we would like the source code to be accessible for anyone who is interested, to provide feedback, to further develop it or to modify it in order to fit certain individual needs. These aspects are best covered by the open-source software model.\\

Finally, we want to emphasize on the need for this simulator, relating to the need for simulators in Ubicomp as identified by Reynolds, Cahill and Senart in \cite{reynolds2006requirements}: ''The use of simulation technology in ubiquitous computing is of particular importance to developers and researchers alike. Many of the required hardware technologies such as cheap reliable sensors are only reaching maturity now, and many of the application scenarios are being designed with the future in mind and well in advance of the hardware actually being available. Furthermore, many of the target scenarios do not lend themselves to onsite testing, in particular, scenarios which require deployment of large numbers of nodes or devices. In addition, simulation enables researchers to evaluate scenarios, applications, protocols and so forth without the difficulties in dealing with hardware sensors and actuators, and also offers greater flexibility since it is easy to run a set of simulations with a range of parameters''.\\

\section{Problem Statement} % (fold)
\label{sec:problem_statement}

Designing and developing mobile context-aware systems in a real-life set-up represents a tedious, costly and time consuming process. In addition, most systems require testing and experimentation in their target location for valid and beneficial results to be produced; however it is difficult and costly to set up a certain location to support the designed system. Moreover, incorporating physical objects into the system design, adds up to the aforementioned challenges.

% These facts impose a challenge for researchers to efficiently work with the Egocentric Interaction Paradigm. 

% To overcome these challenges, I am designing and developing a simulation environment for mobile context-aware system design, focusing on the egocentric paradigm. The framework will monitor the agent's context in regards to the world space and the context of surrounding entities.\\

% subsection problem_statement (end)

%************************************************
\section{Goal} % (fold)
\label{sec:goal}
%************************************************
The goal of this project is to design and implement the basis of an open-source simulation environment for mobile context-aware design using Java based technologies. The framework will capture the context of physical objects, virtual objects and mediators \cite{pederson2011situative} in regards to the agent's parameters. The agent represents an avatar of the user interacting with the simulated environment.\\

The following set of goals will guide our work:
\begin{enumerate}
	\item Host all the work in a publicly available source control repository
	\item Follow a modular design to implement the core of the framework as a set of decoupled components:
		\begin{itemize}
			\item The \emph{Data Model}. This represents the contextual entities and the relationship between them that the simulator will monitor, briefly detailed in Section \ref{sec:data_model}.
			\item The \emph{World Model}. Represents the current status of the simulated environment (e.g. position of the agent, status of a certain switch etc.). The structure of the entities in this model have been defined in the Data Model.
			\item A universal communication protocol to access the core. It needs to be universal so that other components, can easily be developed and integrated.
			\item Query and modify API of information in the World Model. To make the integration between the core of the framework and the other modules, these APIs will need to provide a complete set of methods.
		\end{itemize}
	\item Evaluate the framework:
		\begin{itemize}
			\item Implement a module to modify the state of the simulated environment (DummySim). This comprises of manipulating the agent's parameters and performing actions like picking up an item, touching a screen etc. The purpose of this module is to evaluate the modify API of the core. It also servers as a tool to modify the environment which is a required input for the SSM model to be evaluated.
			\item Implement the \emph{SSM Model}. Using the current World Model's query API, retrieve the current set of contextual data and apply the rules defined in the SSM. This component will actively monitor the current status of the simulator, having as a result an up to date collection of SSM entity sets: world space, perception space, recognizable set, examinable set, action space, selected set and manipulated set. This evaluation will be considered successful if the results are similar to the results obtained in the initial evaluation of the SSM described in \cite{pederson2011situative}.
		\end{itemize}
\end{enumerate}

The followings are not target of this work:
\begin{itemize}
	\item Create a close-to-real world interaction. Projects that simulate real world scenarios, employ game alike solution providing complex graphics and interaction between the user of the simulator and the simulated world. This can be viewed as a future work; in the current work I am focusing on the framework's core.
\end{itemize}

\subsection{Data Model}\label{sec:data_model}
The data model is aimed at defining the supported entities and the relationships between them. In this work the aim is to support physical objects, virtual objects and mediators.\\

The agent represents a virtual character that can freely moved around in the environment. Beside the position the of the agent, the user can also control more complex parameters of the agent like orientation (affects field of vision), sitting down / standing, hands orientation etc. The agent will be also be able to carry out a certain set of actions like picking objects up, putting objects down etc.\\

Physical objects will be described as entities with a certain set of properties (location, weight etc.). Entity composition will be supported to some extend: place objects on top of each other (i.e. a stapler on a table), a mobile phone in the backpack worn by the agent.\\

This section created a brief overview of the supported entity types and relationships. They will be described in details in Chapter \ref{ch:design}.
% subsection goal (end)

%************************************************
\section{Method} % (fold)
\label{sec:method}
%************************************************

In order to achieve the goals detailed in Section \ref{sec:goal}, I am going to follow an iterative design process \cite{mackay1997hci}. Each iteration will comprise of two steps:
\begin{enumerate}
	\item prototyping
	\item evaluation
\end{enumerate}

As part of the process, there will be three iterations:
\begin{itemize}
	\item Early prototype. This step represents the skeleton of the system, where I implemented a first version of the simulation framework; \emph{EgoSim}. As part of the implementation, I have set up a trivial test environment which provides input to the underlying framework to run the SSM classifications. The resulting SSM sets are visualized, in real time, in the \emph{ContextClient}. This prototype has been evaluated in close collaboration with the supervisors of this thesis.
	\item High fidelity prototype. Based on the input from the evaluation of the early prototype, I modified the system accordingly. Further, I have implemented all the intended interaction mechanism between the agent and the environment and I have also developed a public API to provide easy access to the SSM spaces for third party services. As part of this prototype, I have designed an advanced simulation environment based on a real-life inspired scenario. This prototype has been evaluated by the target group of 17 participants. The purpose of the evaluation was to test the usability of the simulator, the responsiveness of the ContextClient, the correctness and usability of the SSM sets, as well as how intuitive is the interaction with the environment.
	\item Final product. In this final iteration I have fixed the bugs discovered during the previous iteration. Furthermore, I have designed a comprehensive documentation on how to use the EgoSim framework. For this iteration I have created a third simulation environment and, this time, the 17 participants have evaluated the usefulness and ease-of-use of the EgoSim framework, by setting up a simulation based on the framework and the documentation. As most participants were not experts of the Egocentric Interaction Paradigm, I have also evaluated framework as an educational tool by comparing the participants understanding of SSM related concepts before and after the evaluation.
\end{itemize}

I will wrap up our research by drawing conclusion based on our findings and establishing goals for future work.

% subsection method (end)

%************************************************
\section{Scenario} % (fold)
\label{sec:scenario}
%************************************************
To better visualize the importance of the EgoSim framework, in this section we present a concrete scenario which has been used during the framework's evaluation. The labels in the scenario's description will be referenced in the requirements Section \ref{sec:requirements}.\\

The hypothetical problem presented in this scenario is that families cannot make their homes secure enough for their children. To provide a solution for this problem, there's a need for a system to constantly monitor the objects a child should keep away from and should not be interacting with. Based on this context information, we can further build various software service to secure the house from the child's actions. One example of such service is the "Secure Outlets Service" (SOS). This service has to detect whenever a child is approaching an outlet with a small object, in which case it should shut down the electricity switch for outlets, preventing the child from the possibility of getting electrocuted.\\

A good solution for this problem could be based on the egocentric interaction paradigm, as it can categorizes all the objects of interest around the human agent. Before implementing the system in a real world set-up, it is best to validate the design by simulating it. To visualize how the EgoSim framework could help, this is how we imagine a researcher would use it:

\begin{flushright}{\slshape
The researcher starts out by setting up the simulation, using the EgoSim framework. First, she/he identifies the target environment -- it resembles a living room where the child spends most of the time. Next, she/he 
needs a virtual model of the environment \textlabel{(1)}{scenario:1}. The model is populated \textlabel{(1A)}{scenario:1A} with everyday physical objects \textlabel{(1B)}{scenario:1B} and devices \textlabel{(1C)}{scenario:1C}. Some of the simulated entities have to be constantly monitored \textlabel{(1D)}{scenario:1D} by the system. Moreover, some of the entities should have the option to be picked-up \textlabel{(1E)}{scenario:1E} and put-down \textlabel{(1F)}{scenario:1F} onto surfaces \textlabel{(1G)}{scenario:1G} (e.g. a table, the floor, etc). Optionally, picked-up entities could interact with other entities \textlabel{(1H)}{scenario:1H}, other than surfaces. Finally, the monitored entities need to provide the necessary configuration information required by SSM model in the egocentric interaction paradigm \textlabel{(1I)}{scenario:1I}.
} \\ \medskip
\end{flushright}

\begin{flushright}{\slshape
After the researcher has finished setting up the simulation, using the EgoSim framework, she/he runs the simulation in order to test out the system design. Once the simulation is started up, the researcher finds himself impersonating a child, the entity this system is designed for. The child is represented by a virtual avatar \textlabel{(2)}{scenario:2}, placed within the simulated target environment, in this case the living room. The avatar can be moved around the living room \textlabel{(2A)}{scenario:2A} and can look around to inspect the surroundings \textlabel{(2B)}{scenario:2B}. If the agent is close enough to an entity, it can be interacted with (i.e. picked-up, put-down, etc) \textlabel{(2C)}{scenario:2C}. As any of the previous actions are carried out, the SSM monitoring service \textlabel{(2D)}{scenario:2D} keeps track of the agent's position as well as the distance to each object in the agent's visual spectrum (what can be seen at a certain moment in time). Based on this information, the entities are being categorised, in real time, into SSM sets \textlabel{(2E)}{scenario:2E}.
} \\ \medskip
\end{flushright}

\begin{flushright}{\slshape
The Secure Outlets Service is implemented as a third party service, listening to changes in the SSM sets through the EgoSim's API provided by the running simulation. As the simulation unfolds, the researcher controls the avatar to pick up a pen located on the living room table. With the pen picked up, the agent is controlled to approach one of the simulated outlets; if close enough, the service detects an immediate possible threat, as there is a high possibility the child could insert the pen into the outlet. The service shuts down the electricity switch for outlets, preventing the child from the possibility of getting electrocuted \textlabel{(3)}{scenario:3}.
} \\ \medskip
\end{flushright}

The points in the above scenario marked with (1-3) will help motivating the high level requirements we will come up with for EgoSim in Section \ref{sec:requirements}.
% subsection scenario (end)

\section{Thesis Overview} % (fold)
\label{sub:thesis_overview}

The rest of the work is structured as follows: Chapter \ref{ch:related_work} reviews similar systems, with respect to the criteria defined in the current work.\\

Chapter \ref{ch:design} and \ref{ch:implementation} constitute the core of the thesis. Chapter \ref{ch:design} presents the design of the system, while Chapter \ref{ch:implementation} discusses the implementation details of the practical work.\\

Chapter \ref{ch:evaluation} is dedicated to the evaluation of EgoSim. Chapter \ref{ch:conclusion} summarizes the contributions brought by the this thesis and presents the main goals of our future work.

% subsection thesis_overview (end)