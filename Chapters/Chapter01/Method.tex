%************************************************
\section{Method} % (fold)
\label{sec:method}
%************************************************

In order to achieve the goals detailed in Section \ref{sec:goal}, we are going to follow an iterative design process \cite{mackay1997hci}. Each iteration will comprise of two steps:
\begin{enumerate}
	\item prototyping
	\item evaluation
\end{enumerate}

As part of the process, there will be three iterations:
\begin{itemize}
	\item Early prototype. This step represents the skeleton of the system, where we implemented a first version of the simulation framework; \emph{EgoSim}. As part of the implementation, we have set up a trivial test environment which provides input to the underlying framework to run the SSM classifications. The resulting SSM sets are visualized, in real time, in the \emph{ContextClient}. This prototype has been evaluated in close collaboration with the supervisors of this thesis.
	\item High fidelity prototype. Based on the input from the evaluation of the early prototype, we modified the system accordingly. Further, we have implemented all the intended interaction mechanism between the agent and the environment and we have also developed a public API to provide easy access to the SSM spaces for third party services. As part of this prototype, we have designed an advanced simulation environment based on a real-life inspired scenario. This prototype has been evaluated by the target group of 17 participants. The purpose of the evaluation was to test the usability of the simulator, the responsiveness of the ContextClient, the correctness and usability of the SSM sets, as well as how intuitive is the interaction with the environment.
	\item Final product. In this final iteration we have fixed the bugs discovered during the previous iteration. Furthermore, we have designed a comprehensive documentation on how to use the EgoSim framework. For this iteration we have created a third simulation environment and, this time, the 17 participants have evaluated the usefulness and ease-of-use of the EgoSim framework, by setting up a simulation based on the framework and the documentation. As most participants were not experts of the Egocentric Interaction Paradigm, we have also evaluated framework as an educational tool by comparing the participants understanding of SSM related concepts before and after the evaluation.
\end{itemize}

We will wrap up our research by drawing conclusion based on our findings and establishing goals for future work.

% subsection method (end)