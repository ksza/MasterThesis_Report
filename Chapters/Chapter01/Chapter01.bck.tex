%************************************************
\chapter{Introduction}\label{ch:introduction}
%************************************************

%I am structuring the introduction according to this guide: http://pages.cpsc.ucalgary.ca/~saul/wiki/pmwiki.php/Chapter1/Deconstruction. It's Saul Greenbarg's guide to thesis chapter 1's, but I think it works fine for smaller projects as well. If we find %that the structure doesn't work we can always change it, but for now I think its a good way to do it (as it forces me think about every aspect)

Mobile Context-Aware Systems are hard and costly to develop. Engineering challenges include a) the need to identify and deploy suitable sensors to feed into a context model, b) to design a context model that is able to determine the mobile agent's context with adequate accuracy. An additional challenge is to determine a suitable mobile interaction paradigm (interaction modalities, timing with respect to ongoing real world activity, etc.) that allows the human agent to inspect and modify the behaviour of the context-aware system. \\

This master thesis project will develop a simulator of physical environments intended to facilitate the design of future context aware systems. Apart from generating data that is relatively easily acquired and used in context-aware models and systems today (time, agent location, etc.) the simulator will also provide data which up until now has been too challenging to capture using sensors available today such as the location (relative, absolute) and state of physical non-computational everyday objects. The simulator will also be able to provide state information about computational devices such as mobile phones and desktop PCs (e.g. what applications that are currently running/displayed on the device). \\

The simulator will make it possible for the context-aware system designer to import or design physical environments (e.g. buildings, rooms) and populate those environments with everyday objects (chairs, tables, books, coffee cups) as well as a few computational devices (PCs, mobile phones, digital displays etc.). Once the environment has been designed, a participant in an experiment can walk around in the simulated environment and interact with objects. The behavior of the experiment participant will generate context data which the simulator feeds to context modeling components as well as high-level services that can then act on that information and for instance display information on displays situated in the simulated environment whenever a service requests to contact the human agent. \\

The main goal of this work is to research and implement an open source environment simulator that can accurately capture various context data, which can easily be consumed by services of the simulator (context-aware systems). There have been many approaches towards providing a solution for this topic, but the novelty in our approach is twofold. On one hand we are aiming at developing an open source. On the other hand we aim at capturing a large variety of context data, allowing for a fine grained, almost real-life, inspection of the simulated environment. To further clarify, we are diving deaper into the system and describe the goals of each module as a set of requirements/features.\\

At the core of the system lies the \emph{World Model} made up by the set of all simulated entities and their context information. The simulated entity types are: physical objects, virtual objects and mediators, and the simulated agent (person). \emph{Physical objects} are everyday objects like a chair, table, door, cup, sofa etc. \emph{Virtual objects and mediators} are basically software entities (virtual objects) that can be interacted with through \emph{mediators} (input/output devices). For instance, the music player on a mobile device is a virutal object, while the mobile device is a mediator offering two components for I/O: the screen (both input and output, as the agent can both view and change the state of the player) and the speakers (output only through which the agent can percieve the player) which is both an input and output \cite{pederson2011situative}. All the previously mentioned entities can be in direct relation with the \emph{Agent}, mutually affecting each others context.\\

The world model offers an interface that helps to modify the state of entities or quiry about the state of the entities. Making use of these interfaces, we define two higher level modules: the \emph{Context Model} and the \emph{Graphical Simulator}.\\

The Context Model represents the \"current situation\", a computed state of the system in t0 based on a set of rules. An example of a context model is the distance of the agent relative to a chair in meters. At every moment in time, the chair and the agent will have certain value for their location data. Based on simple arithmetics, the context model computes the distance in meters from the user to the chair. As the context data in the world model changes, the context model updates reflecting a real-time model of the simulated environment.\\

The Graphical Simulator represents the GUI of the simulated environment. It allows us to visually percieve the simulated environment and to interact with it through the Agent. We see the Agent as our avatar in the virtual world empowering the user of the simulator to freely walk arround and interact with the other entities.\\

Next, detailed goals of our system.

\paragraph{What do we want to simulate?} Our aim is to simulate a multiple room apartment.

\paragraph{What context information are we gathering?} We monitor both a set of general context and a set of entity specific context data. In the general cathegory we have: location, time etc. As specific context:
\begin{enumerate}
	\item physical objects 
\end{enumerate}

The quality of the simulator will be assessed by implementing a specific context model, the Situative Space Model \cite{pederson2011situative} and connecting it to the system, and let a simple context-aware service (e.g. a reminder application) react on the contextual changes as interpreted by the model. \\

\section{Problem Statement} % (fold)
\label{sec:problem_statement}

Designing and developing mobile context-aware systems in a real-life set-up represents a tedious, costly and time consuming process. In addition, most systems require testing and experimentation in their target location for valid and beneficial results to be produced; however it is difficult and costly to set up a certain location to support the designed system. Moreover, incorporating physical objects into the system design, adds up to the aforementioned challenges.

% These facts impose a challenge for researchers to efficiently work with the Egocentric Interaction Paradigm. 

% To overcome these challenges, I am designing and developing a simulation environment for mobile context-aware system design, focusing on the egocentric paradigm. The framework will monitor the agent's context in regards to the world space and the context of surrounding entities.\\

% subsection problem_statement (end)

%************************************************
\section{Goal} % (fold)
\label{sec:goal}
%************************************************
The goal of this project is to design and implement the basis of an open-source simulation environment for mobile context-aware design using Java based technologies. The framework will capture the context of physical objects, virtual objects and mediators \cite{pederson2011situative} in regards to the agent's parameters. The agent represents an avatar of the user interacting with the simulated environment.\\

The following set of goals will guide our work:
\begin{enumerate}
	\item Host all the work in a publicly available source control repository
	\item Follow a modular design to implement the core of the framework as a set of decoupled components:
		\begin{itemize}
			\item The \emph{Data Model}. This represents the contextual entities and the relationship between them that the simulator will monitor, briefly detailed in Section \ref{sec:data_model}.
			\item The \emph{World Model}. Represents the current status of the simulated environment (e.g. position of the agent, status of a certain switch etc.). The structure of the entities in this model have been defined in the Data Model.
			\item A universal communication protocol to access the core. It needs to be universal so that other components, can easily be developed and integrated.
			\item Query and modify API of information in the World Model. To make the integration between the core of the framework and the other modules, these APIs will need to provide a complete set of methods.
		\end{itemize}
	\item Evaluate the framework:
		\begin{itemize}
			\item Implement a module to modify the state of the simulated environment (DummySim). This comprises of manipulating the agent's parameters and performing actions like picking up an item, touching a screen etc. The purpose of this module is to evaluate the modify API of the core. It also servers as a tool to modify the environment which is a required input for the SSM model to be evaluated.
			\item Implement the \emph{SSM Model}. Using the current World Model's query API, retrieve the current set of contextual data and apply the rules defined in the SSM. This component will actively monitor the current status of the simulator, having as a result an up to date collection of SSM entity sets: world space, perception space, recognizable set, examinable set, action space, selected set and manipulated set. This evaluation will be considered successful if the results are similar to the results obtained in the initial evaluation of the SSM described in \cite{pederson2011situative}.
		\end{itemize}
\end{enumerate}

The followings are not target of this work:
\begin{itemize}
	\item Create a close-to-real world interaction. Projects that simulate real world scenarios, employ game alike solution providing complex graphics and interaction between the user of the simulator and the simulated world. This can be viewed as a future work; in the current work I am focusing on the framework's core.
\end{itemize}

\subsection{Data Model}\label{sec:data_model}
The data model is aimed at defining the supported entities and the relationships between them. In this work the aim is to support physical objects, virtual objects and mediators.\\

The agent represents a virtual character that can freely moved around in the environment. Beside the position the of the agent, the user can also control more complex parameters of the agent like orientation (affects field of vision), sitting down / standing, hands orientation etc. The agent will be also be able to carry out a certain set of actions like picking objects up, putting objects down etc.\\

Physical objects will be described as entities with a certain set of properties (location, weight etc.). Entity composition will be supported to some extend: place objects on top of each other (i.e. a stapler on a table), a mobile phone in the backpack worn by the agent.\\

This section created a brief overview of the supported entity types and relationships. They will be described in details in Chapter \ref{ch:design}.
% subsection goal (end)

%************************************************
\section{Method} % (fold)
\label{sec:method}
%************************************************

In order to achieve the goals detailed in Section \ref{sec:goal}, I am going to follow an iterative design process \cite{mackay1997hci}. Each iteration will comprise of two steps:
\begin{enumerate}
	\item prototyping
	\item evaluation
\end{enumerate}

As part of the process, there will be three iterations:
\begin{itemize}
	\item Early prototype. This step represents the skeleton of the system, where I implemented a first version of the simulation framework; \emph{EgoSim}. As part of the implementation, I have set up a trivial test environment which provides input to the underlying framework to run the SSM classifications. The resulting SSM sets are visualized, in real time, in the \emph{ContextClient}. This prototype has been evaluated in close collaboration with the supervisors of this thesis.
	\item High fidelity prototype. Based on the input from the evaluation of the early prototype, I modified the system accordingly. Further, I have implemented all the intended interaction mechanism between the agent and the environment and I have also developed a public API to provide easy access to the SSM spaces for third party services. As part of this prototype, I have designed an advanced simulation environment based on a real-life inspired scenario. This prototype has been evaluated by the target group of 17 participants. The purpose of the evaluation was to test the usability of the simulator, the responsiveness of the ContextClient, the correctness and usability of the SSM sets, as well as how intuitive is the interaction with the environment.
	\item Final product. In this final iteration I have fixed the bugs discovered during the previous iteration. Furthermore, I have designed a comprehensive documentation on how to use the EgoSim framework. For this iteration I have created a third simulation environment and, this time, the 17 participants have evaluated the usefulness and ease-of-use of the EgoSim framework, by setting up a simulation based on the framework and the documentation. As most participants were not experts of the Egocentric Interaction Paradigm, I have also evaluated framework as an educational tool by comparing the participants understanding of SSM related concepts before and after the evaluation.
\end{itemize}

I will wrap up our research by drawing conclusion based on our findings and establishing goals for future work.

% subsection method (end)

\section{Thesis Overview} % (fold)
\label{sub:thesis_overview}

The rest of the work is structured as follows: Chapter \ref{ch:related_work} reviews similar systems, with respect to the criteria defined in the current work.\\

Chapter \ref{ch:design} and \ref{ch:implementation} constitute the core of the thesis. Chapter \ref{ch:design} presents the design of the system, while Chapter \ref{ch:implementation} discusses the implementation details of the practical work.\\

Chapter \ref{ch:evaluation} is dedicated to the evaluation of EgoSim. Chapter \ref{ch:conclusion} summarizes the contributions brought by the this thesis and presents the main goals of our future work.

% subsection thesis_overview (end)

