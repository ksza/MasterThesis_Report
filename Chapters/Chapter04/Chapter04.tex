%************************************************
\chapter{Implementation}\label{ch:implementation}
%************************************************

% THIS IS FROM UBIWISE. I CAN PROVIDE THIS INFO, OF HOW TO USE THE FINAL PRODUCT IN AN APPENDIX!

% "The core of the simulation is creating the virtual counterparts of devices and the
% picture frames present in the physical environment. The next section charts the path
% taken to create these virtual objects, focusing on the camera as our example. It also
% introduces and develops the two views and their roles separately. We try to give a
% sense of the time a researcher will need for the various steps. We wrap up the
% discussion with a description of the simulator at runtime and we describe how the two
% views come together to form a complete whole."







% Objects within the view frustum are not always on the direction of the camera, so based on the two positional vector, we need to determine the directional vector (the direction of the ray). The directional vector between two positional vectors is given by their difference. Therefore, to find the directional vector from the agent to the centre of the target object, we can subtract the agent's positional vector from the object's positional vector.