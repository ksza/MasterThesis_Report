%************************************************
\section{Participants} % (fold)
\label{sec:eval_participants}
%************************************************
In order to collect relevant technical feedback, we have aimed at recruiting participant having a fair amount of experience in software development. We have recruited a total of 17 participants. 15 of them are Software Engineers, 1 Post-Doctoral Researcher and 1 Associate Professor. In the beginning of the study they were asked to assess their experience in software engineering and, particularly, in mobile development. Moreover, we gathered information on their experience with context-aware computing and the egocentric interaction paradigm. We briefly present the statistics of our participants profile bellow:
\begin{itemize}
	\item Years of experience in Software Engineering -- an average of 6.41 (ranging from 3 - 15)
	\item Years of experience in Mobile Development -- an average of 1.56 (ranging from 0 - 5)
	\item Familiar with concepts Context-Aware Computing -- Yes: 47.1\% (8), No: 52.9\% (9)
	\item Has worked on context-aware system (e.g. a mobile APP using the device's location to perform implicit actions) -- Yes: 29.4\% (5), No: 70.6\% (12)
	\item Familiar with the egocentric interaction paradigm or the SSM -- Yes: 5.9\% (1), No: 94.1\% (16)
\end{itemize}

Analysing the profile of our participants, we can deduct that they have a good seniority level in software engineering and a fairly good knowledge of mobile development concepts. Almost non of them were aware of the egocentric interaction paradigm, but we were able to easily explain the concepts. Learning about the egocentric-interaction paradigm was made easy due to the vast experience the participants have in the field of software engineering.
% section sec:eval_participants (end)