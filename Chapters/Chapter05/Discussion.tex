%************************************************
\section{Discussion} % (fold)
\label{sec:eval_discussion}
%************************************************
The evaluation results can be found in Appendix \ref{sec:res_alf_task} and Appendix \ref{sec:res_childproof_task}. Before we carry on with the discussion, we would like to point out one more time that the level of experience in the field of software engineering of all of our participants was high. We believe that the results of the evaluation provide good guidelines for future work on the system.\\

For the ALF evaluation scenario we came up with a hypothetical problem and provided a solution based on the SSM. We were concerned that the participants might not agree that designing the solution based on the egocentric interaction paradigm is the right approach. Therefore, we have asked them \ref{question1:6} whether the SSM is the right context model for the ALF system and for the third party services developed around it. All the participants agreed that this was the right approach, assuring that we have chosen a strong and relevant scenario for this evaluation.\\

%************************************************
\subsection{Characteristics} % (fold)
\label{sec:eval_characteristics}
%************************************************
The tool we have built is meant to be used by system designers to create simulations. The resulting systems are then used by simulation users and third party services. To meaningfully assess the quality of the EgoSim framework and of the simulations which can be built with it, we have to take a step back to inspect the goals \ref{sec:goal} and and requirements \ref{sec:requirements}. The requirements have been identified based on the goals of the system and guided by an end-to-end scenario \ref{sec:scenario} we have imagined as the intended way to use our system. Given that the result of our work is an interactive system, we have evaluated the system from the end user's point of view.\\

To quantify the user feedback into a meaningful discussion, we have extracted four characteristics of interest which emerged from the goals and requirements of our system. To evaluate how useful our requirements are and to what extent our system have met them, we will look into the \emph{usefulness} and \emph{usability} of both the framework and simulations.\\

Given that simulations created with EgoSim are interactive systems, we are interested about how \emph{responsive} they were found by simulation users. Moreover, the ContextClient is meant to be a close to real-time context visualization tool, therefore we are assessing its responsiveness as well.\\

The core of a simulation is the classification process and the SSM context model. We have found it imperative to assess the quality and relevance of how objects around the virtual agent got classified during the simulation.\\

To conclude, in the remainder of this section we will discuss the evaluation results in the light of four characteristics: usefulness, usability, responsiveness and classification.\\
% section sec:eval_characteristics (end)

%************************************************
\subsection{Usefulness} % (fold)
\label{sec:eval_usefulness}
%************************************************
One of the main objectives of our evaluation was to investigate if the users found our system useful. On one hand we were interested if the participants found simulations built using EgoSim as being useful; on the other hand if they have found the system as a whole useful to simulate systems designed around the SSM context model.\\

%************************************************
\subsubsection{The Framework} % (fold)
%************************************************
All the participants have seen the value of simulating a system using EgoSim, over directly implementing the system into a real set-up. This strengthens the hypothesis of our work, that there is a real need for the system we have built. As one of the users stated in the feedback for question \ref{question1:7}: ''The simulation could provide a cost effective way of gaining valuable insight into the design of an Assisted Living Facility before such a facility is constructed''. Moreover, some participants pointed out some benefits of the simulator we have not stated:
\begin{itemize}
	\item lower costs in case of trial and error
	\item gives you a perception of how people interact with the objects
	\item faster feedback loop
	\item a simulation is a reliable and cheap proof of concept
	\item it allows the discovery of edge cases
\end{itemize}

To further assess the usefulness of EgoSim, all the users found it useful to simulate systems build with the SSM context model at their core. One of the participants stated in the feedback for question \ref{question2:2} ''Objects can be easily decorated with context information and interacting with them can be tested quickly because the framework takes care of basic interaction and movement in the space''. The fact that we have inferred functionality from based on the Ego Metadata configuration, makes the framework a lot more useful for users than we though in the first place. None but one of the participants were previously aware of the SSM context model, but they have managed to successfully build a simulation in the second task: ''with a few easy steps I was able to model the desired environment''.\\

Our system proved to be a useful educational tool as well. Before staring the evaluation, only one participants reported as being familiar with the egocentric interaction paradigm \ref{question1:0}. After the evaluation the participants have reported an average of 4.5 level of understanding about the SSM, on a scale from 1 to 7 \ref{question2:0}. This means using the EgoSim framework, participants got rather good insights into what egocentric integration paradigm means and how the SSM works. In the future, it might prove to be a useful tool for students studying ubiquitous computing to learn about these concepts.\\

In one of the feedback questions we have asked the participants to imagine various system they could build using based on the SSM Sets and the outcome of our classification algorithms. The have came up with a list of interesting ideas \ref{question2:10}; we have listed some of them bellow:
\begin{itemize}
	\item Assistance for people with disabilities (especially the visually impaired), to offer an augmented representation of their environment.
	\item A tourist map that is context aware and can recommend places to visit that are near the current location, that are for example still open (maybe it's night, you don't want recommendations for a zoo), etc.
	\item Security application: auto-lock my computer screen when it is in my Action Space.
	\item Automatic light switching based on the position of the inhabitant of a home (not motion-based). This can be generalized to a system that defines custom operation modes of devices in a home, depending on what the user is doing.
\end{itemize}

All of these systems/services are useful systems and they can easily be simulated with EgoSim. The fact that so many participants were able to come up with ideas of systems build on top of the SSM context model, means the context model has a lot of potential for context aware system development. Therefore, the support EgoSim offers for building simulations with this context model as a cornerstone, based on the user feedback, turns out to be useful based on the user feedback.

%************************************************
\subsubsection{The Simulation} % (fold)
%************************************************
When creating a simulation of a real life environment, immersion of the user into the virtual environment is rather important. The user must identify herself/himself with the agent and the simulated environment must resemble the real environment it actually represents. This has two implications: on one hand the 3D model of the environment must be good enough to resemble the real life setup of the simulated environment; on the other hand, when the simulation is running and the user is interacting with the environment, she/he must feel absorbed to a certain extent. All the participants found the ALF simulation a good simulation for a home \ref{question1:1}. We have build both the ALF and the Childproof 3D models with a third party software and used them with our framework. Moreover, the participants have experienced the virtual environment by means of the running the simulation. The fact that all participants found it a good simulation of a home, assures that third party software can supply useful 3D models. The feedback from our participants makes us believe that we made the right decision to use a game engine to render the 3D models and provide interaction capabilities.\\

We believed the level of object details for this evaluation was not relevant. One of the participants proved us wrong saying ''given the high system requirements I would have expected a better graphics/game engine''. Reflecting upon the issues this is not a game engine problem. The jMonkey Engine has all the capabilities of rendering high quality models. We have simply implemented models with low details. However, developing high detail models requires much more time invested in the modelling process.\\
% section sec:eval_usefulness (end)

%************************************************
\subsection{Usability} % (fold)
\label{sec:eval_usability}
%************************************************

%************************************************
\subsubsection{The Framework} % (fold)
%************************************************
All participants were positive about the usability of the system. Creating a simulation using EgoSim comes down to do some configuration work on top of a 3D model. Using EgoSim to design a new simulations, the participants have reported that identifying and configuring the objects to be classified during simulation, on a scale from 1 to 7 (Very easy - Very hard) was an average of 2.1 \ref{question2:3}. This means that almost everyone has found the process of designing a new simulation easy. We have based this process on the jMonkey SDK's components. We were sceptical at first and thought it might be a laborious and hard to understand process, but based on the user feedback, we conclude it was the right decision.\\

Although, one of the participants was not able to finish the second task \ref{question2:1} due to not being able to augment the desired object models with EgocentricContextData. This was most probably due to the fact that when configuring an object in the jMonkey scene composer, the name of the custom user data must be EGOCENTRIC\_CONTEXT\_DATA. If it is not exactly this, even if the name has an extra space, the framework will fail to identify object to be classified. A user feedback to improve this feature was to ''trim the context variable for space EGOCENTRIC\_CONTEXT\_DATA''. We could indeed rephrase the name into something shorter, but this is a task for future work.\\

%************************************************
\subsubsection{The Simulation} % (fold)
%************************************************
When running the simulation, the participants felt that interacting with the environment was very intuitive. On a scale from 1 to 7 (Not intuitive at all -- Highly intuitive), they have reported an average of 5.5 \ref{question1:3}. This translates into the interaction with the environment being very intuitive. This is of course valid only for moving around the environment and for the default interaction we have provided for objects -- pick-up/drop-down. Some users felt that a generic USE action would be more appropriate ''A generic USE action would help is many situations as many objects that are interacted with might have a main USE method. pickup/putdown could be specific implementations of use''. Moreover, other kind of interactions have been requests like turn off/on for electronics, shower, etc. All these aspects will be discussed in Section \ref{sec:future_work}.\\

The collision detection (the fact the agent could not pass through objects) was especially well received ''I think moving around was very close to reality and I liked the fact that it was aware of the obstacles. For example, you could not just move forward through the couch to get to the coffee table''.\\

%************************************************
\subsubsection{Context Data and Contextual Interactions} % (fold)
%************************************************
The custom interactions we have implemented in the fist scenario were also well received: ''The Piano playing actions were nice''. Some participants noted that support for building custom interaction would have been useful: ''Would be nice if people could declare their own type of interactions''. This is actually possible with the current implementation, unfortunately we could not add a tasks for the participants to try this out; the evaluation took a fair amount of time as it is. The fact that it is a required feature, means we have well anticipated this need by implementing support for both CUSTOM interaction with objects, as detailed in Listing \ref{lst:custom_interaction}, and COMBINED interaction between two objects, as detailed in Listing \ref{lst:custom_interaction}.\\

Most context aware systems need to access the historical information of changes in context of various entities. We have not implemented this aspect, but we should plan for this in future work. One of the participants even mentioned the lack of this feature stating that it would be useful to have ''Historical information about the agent: what actions has he performed so far, objects with which he has interacted with so far, etc''.\\

A potential weakness of the framework was pointed out by one of the participants saying ''a potential weakness of this framework would be the lack of information regarding the relationship between objects''. In the current implementation we monitor the context of surrounding objects only from the agent's perspective. But context information that is affected by relationship between objects might also be relevant. This could be solved by adding support for sensors to our framework, which we lack at the moment.\\

%************************************************
\subsubsection{The API} % (fold)
%************************************************
To evaluate the framework from a third party service's point of view, in the second task we have given the participants to solve a hypothetical problem in the context of the scenario, as a developer of a third party service based on data available through the API. 16 out of the 17 participants were able to provide the correct solution \ref{question2:10}. This makes us believe the data in the SSM sets provided through the API is explicit enough to build service aware of the current context in the ongoing simulation.\\

In the current implementation context changes occur only as the agent interacts with the environment. Some participants pointed out that it would be useful to modify some context data through the API. We will reconsider this idea in future work.\\

Overall, the framework as a whole was perceived by the participants as being highly usable \ref{question2:4}. One of the participants even said that ''it is very easy to import a scene and augment the objects with different information and interaction types''.\\
% section sec:eval_usability (end)

%************************************************
\subsection{Responsiveness} % (fold)
\label{sec:eval_responsiveness}
%************************************************
Target of our evaluation was also to assess how responsive the simulation and the context client turned out to be. The simulation was found to be responsive and running smoothly by 88.2\% of the participants \ref{question1:8}. We are especially happy about this outcome because the classification process implies heavy computation under the hood, which means we have done the right thing implementing the classification as a standalone process. One of the users reported that ''there was a glitch when I could not move forward when getting out of bed. I had to move in a different direction and after that come back and approach the closet''. This is most probably because we have used a deprecated CharacterControl\footnote{\url{http://hub.jmonkeyengine.org/javadoc/com/jme3/bullet/control/CharacterControl.html}} class from the jMonkey framework to help us in with the agent's movement. In future work we should address this issues by upgrading to the BetterCharacterControl\footnote{\url{http://hub.jmonkeyengine.org/javadoc/com/jme3/bullet/control/BetterCharacterControl.html}} class.\\

When asked about the responsiveness of the context client \ref{question1:5}, 76.5\% of the participants reported it being responsive and updating the context changes close to real time. Some of the comments from the other 23.5\% made us realised that there's place for improvement for the context client. Currently the context client is implement as a web page served from within the running simulation. The page automatically updates every second. This approach has some drawbacks as it will not display the context changes instantaneously. Moreover, it queries all the SSM sets over and over again without some of them modifying at all. This creates an unnecessary overhead as the context client is accessing the same shared resource the API and classification modules are using. Therefore, a more suitable implementation for the context client would be based on web sockets. This way, we would reverse the flow of information; the server would send notifications to the opened web page every time a change in context occurs. Anyhow, for the purpose of this evaluation the context client played out well and it's responsiveness was well received.
% section sec:eval_responsiveness (end)

%************************************************
\subsection{Classification} % (fold)
\label{sec:eval_classification}
%************************************************
A little more than half of the users have reported that the monitored objects were correctly classified into SSM sets \ref{question1:2}. This means our classification algorithm performed decently. The rest of the participants reported there were cases when the classification did not work as expected. There were two problems the users have noticed which prevented the classification to work as expected.\\

The first problem was that some objects were not being classified at all. These objects were actually never augmented with egocentric data, therefore with the current implementation they cannot be classified. We though this was clear from the description in the evaluation, but we might have had to stress more on this matter. However, if all the objects in a scene were to be augmented with Ego metadata, during simulation they would all be taken into account and classified.\\

The second problem was that although some objects were visible, they were not classified as being perceived by the agent. This is a problem we are aware of and plan on addressing in future work. At the heart of our classification algorithm lies the way we determine if an object is currently visible to the agent or not. To do this, we cast a ray from the agent's current location to the centre point of the object's bounding box (think of it a the object's centre of gravity). If the ray intersects any other object before hitting the middle of the object, we consider the object as not being visible to the agent. This works in most cases, as the user feedback shows. But it has a lot of edge cases. For example the centre of a table can be occluded by a coin; this does not make the table unrecognisable, it just make our algorithm generate a false positive. Likewise, the agent could see through a hole in the wall the only centre of a drawer. Without other details, that object might be unrecognisable.\\

We have foreseen this limitation in Section \ref{sec:monitoring_service} and we have provided a solution for the future work. To overcome these kind of edge cases, we could cast a large number of rays to various points on the object and based on a more advanced algorithm, determine if the object is being occluded. We have decided to leave the design of a more advanced computation for future work, as the current solution is good enough to cover most of the cases, while developing a more advanced and correct algorithm needs more research in the field of 3D modelling and rendering.\\

One of the participants provided a very helpful feedback in this matter ''I consider the condition for an object to enter the perception space to be too weak. A user cannot identify an object if only a pixel is within his field of vision. Humans are built to recognize patterns, therefore a pattern-oriented metric could be used. For instance, the introduction of a diversity factor would help the simulation. If enough diversity of the object is within the perception space, it could trigger its presence in the perception space''. We entirely share this opinion and will be taken into consideration in further improvements of the classification algorithm.\\

For the agent to be able to interact with an object, that objects must be in the ActionSpace, therefore within reach, or within the ActionDistance. We have tailored a default value for this property. The system designer is free to modify this value at design time, but the participants did not modify it. Most of them (88.2\%) have felt that the default value was close to realistic \ref{question1:2}. One participant felt it was too close and another that it was too far. The problem discussed above has impact when classifying into each sets, therefore affecting the interaction between the agent and the surrounding objects.\\

Taking a step back to Section \ref{sec:eval_usability}, the fact that such a large number of participants have found interacting with the surrounding objects highly intuitive, means that they were able to easily pickup/drop down objects. This means the classifications are running close to real time. Why is that? Because every time the field of vision of the agent changes, we run a new classification. Even the slightest change to user's field of vision would trigger a new classification. Within the framework, decision are taken based on the latest available result of the classifications. As the simulation was perceived to run smoothly and interacting with the environment seemed natural, we conclude that decisions during the simulation were made according to the current context, meaning the classifications were execute almost instantaneously. In future work it would be interesting to evaluate how a very large number of tracked objects affect the performance of our classification algorithm.
% section sec:eval_classification (end)

% section sec:eval_discussion (end)