%BEGIN: Service in Smart Homes
\section{Services inside the Smart Home: A Simulation and Visualization tool}\label{sec:services_in_smart_homes}

In this work \cite{lazovik2009services} the aim is to reduce the testing costs of smart homes. The goal was achieved by implementing a simulation and visualization tool which replaces services in a smart home with virtual stubs behaving just like the real hardware installed in a house.\\

For the implementation of virtual stubs they have employed the service-oriented computing paradigm, which is widely used to implement systems requiring high interoperability, scalability, security and reliability. One of the main advantages is that such software can seamlessly integrate other systems written in other programming languages and even deployed under different operating systems.\\

The simulation scenarios are built using Google SketchUp \cite{sketchup:online}. This was extended with a set of tools extending its visual representation of a house with virtual home interactive web services supporting SOAP messages. Further, the visualization component is written as a set of plug-ins for Google SketchUp.\\

The simulation and visualization tool allows to simulate any possible home automation scenario, with the possibility of modeling the user and its interactions with the home.
%END: Service in Smart Homes