%************************************************
\section{Conclusion} % (fold)
\label{sec:conclusion}
%************************************************

% 1) reflection over the whole design process (what would you have done different, knowing what you know now (if anything)
In this thesis, we have designed and developed EgoSim, an open-source simulation environment for mobile context-aware system design. The framework is meant to be used by system designers to produce simulations of their target egocentric system. The resulting simulations are body-centric, meaning that the simulation user interacts with the simulation environment by controlling a virtual avatar. As the agent interacts with the environment, physical objects and devices in its surroundings are classified according to the SSM context model. This data is available through an API to be accessed by third party service and to be visualised in the ContextClient.\\

To help gather the requirements of our system, we have imagined an end-to-end scenario of how the finished system would be used by end users. This scenario was approved of by Thomas Pederson, one of the experts in the egocentric interaction paradigm. Aided by our goals \ref{sec:goal} and requirements \ref{sec:requirements}, we have decided that a modern game engine is the best technological choice to base our design upon. Therefore, the final architecture of our system is based on the architecture of modern game engines. Although we would have liked to follow an iterative design process, due to the high amount of research work and technical tasks required by this project we were constrained on trying to build the right system from the begging; hence, we took on a high risk of failure.\\

As a retrospect of the design process, if we have had more resource available, we would have had followed an iterative design process with 2 iterations as follows. For the first iteration, using the requirements we came up with, design an initial prototype. Evaluate the first prototype with a group of experienced ubiquitous system designers and developers. Using this feedback design a second prototype, the final design of our system.\\

We have implemented our design on top of the jMonkey game engine, which we have found ideal for research purposes as it is: open-source with a large and active community, well documented, purely written in Java and works with 3D models produced by third party modelling tools (i.e. Blender, one of the most popular open source 3D modelling application). Using EgoSim we have developed two prototypes, used in the evaluation process. The first one had a 3D model built pragmatically, for the initial testing phase of our system; the second one was a simulation of an assisted living facility, using a 3D model created with Blender.\\

To evaluate our work, we have recruited 17 participants, all with strong software engineering skills. First they have evaluated the two aforementioned prototypes, followed by the implementation of a simulation from scratch. All participants have found both the framework and the simulations useful, highly usable and responsive. Only a little more than half the participants found the classification process to be reliable. The classification performed well in most cases, but it gives inaccurate results in edge cases. We have provided details on how to improve the classification process in the future work \ref{sec:future_work}. This should be the first improvement taken care of in the near future of the project.\\

We have captured the list of goals in the future work as ''issues'' in the source code repository of our project \url{https://github.com/ksza/EgoSim}. This way, if anyone is willing to contribute to the project, immediate work items are already created and can be addressed.\\

The evaluation process has concluded that we have successfully met the four goals (\ref{goal:1}, \ref{goal:2}, \ref{goal:3}, \ref{goal:4}) we have set out in the begging of this project and that we came up with useful requirements (\ref{subsec:user_stories}, \ref{subsec:nfrs}) which we have successfully implemented into a highly usable system.\\

% As a hypothesis of our work, we strongly believe that physical objects and devices are an essential part of the context of a human agent. Moreover, designing and developing mobile context-aware systems, in general, in a real-life set-up represents a tedious, costly and time consuming process. To help aid system design from this body-centric perspective we have designed and developed a simulation environment, which can be used to easily set up a 


% 2) discussion of the main findings of the evaluation (the evaluation chapter has a more detailed discussion). what does it point towards? where is the future of this kind of systems?
% 3) future work (list stuff that was found in eval which could improve future versions of the system, ideas you had yourself but discarded due to lack of resources, etc.)

% section sec:conclusion (end)