%************************************************
\section{Simulation Designer} % (fold)
\label{sec:simulation_designer}
%************************************************
Creating the environment model is the first step required by our system. As described in requirements \ref{us:1} and \ref{us:2}, the simulation designer needs to empower the researcher to create the model of the target environment and to identify and configure the objects to be monitored.\\

To proceed, first we need to figure out what kind of model we are going to use to represent the environment. Similar projects like DiaSim \ref{sec:diasim} have used a 2D representation, while Tatus \ref{sec:tatus} or UbiWise \ref{sec:ubiwise} have used a 3D representation of the environment. In this context, 2D stands for two-dimensional while 3D stands for three-dimensional.\\

Think from the agent's perspective, a 2D model can cover two dimensions: location and orientation. As mentioned in requirement \ref{us:4}, the simulation user must be able to control the agent's movement (location and orientation) as well as the eyesight (the visual spectrum). The projects which have used a 2D representation were focusing solely on proximity, while the projects aiming for a more realistic interaction have used a 3D model. Such a model can encapsulate more details, proving a better sense of reality as required in \ref{us:5}.\\

With the 3D model in place, the system designer has to identify the objects she/he desires to be monitored and classified during the simulation. To prepare the model as such, we have designed a list of properties each objects will need to carry; some of them are mandatory, while others can have meaningful default values. The properties are detailed in the following list of \textlabel{object properties}{design:object_properties}:
\paragraph{ID} - to uniquely identify an object within the environment. This parameter is mandatory and it is up the system designer to ensure that each object's ID is unique.

\paragraph{Type} - describes the type of the object. We need to support physical objects and devices \ref{us:1}, so this parameter can have two values: Physical or Device.
% section simulation_designer (end)