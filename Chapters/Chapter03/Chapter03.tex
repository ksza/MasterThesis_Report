%************************************************
\chapter{Design}\label{ch:design}
%************************************************
In this section we will present in details the possibilities and choices involved in accommodating the goals of this system. We start out by defining the requirements of the framework. We discuss briefly the the challenges of building onto some of the related systems. Next, we define the overall architecture of the system that can accommodate the requirements, followed by detailing the possibilities and choices for each component in the architecture. We conclude this chapter with the final design of the framework.

%************************************************
\section{Requirements} % (fold)
\label{sec:requirements}
%************************************************
In Section \ref{sec:goal} we have defined the main goals of the system as a list of high level system requirements. They are very generic and hard to establish an architecture upon. Because they do not contain details of the functionality the end-user of the system will be able to use. To have a clear vision of what this system needs to achieve, we will defined in this section a list of requirements from the end-user's perspective.\\

We have decided to use the user stories concept, as a way to express the requirements of this system.\\

Agile Software Development is an iterative and incremental software development methodology, promoting adaptive planning, evolutionary development and delivery, while encouraging rapid and flexible response to change \cite{agile:online}. It is mainly used in the industry. One of the best practices in Agile Software Development for requirements is \emph{user stories}. We have not employed Agile Software Development in this work, but we have used the user stories concept as the agile methodology and the iterative design process resemble similarities and applying the user stories in this context, comes naturally.\\

''A user story describes functionality that will be valuable to either a user or purchaser of a system or software'' \cite{cohn2004user}. User stories are written from the perspective of the system's users. A specific group of users, will use the system in a certain way; the feature requested by this group represents a \emph{user role}. The user role does not necessarily have to be assigned to a human user; it can also be an external service in need of accessing some data.\\

%************************************************
\subsection{User Roles}\label{subsec:user_roles}
%************************************************
Writing the requirements following the user story concept, helps us reflect upon the needs this framework must satisfy, from the user's perspective. Based on the goals of the system defined in Section \ref{sec:goal} and supported by the use case scenario defined in Section \ref{sec:scenario}, we have first identified the user roles for this framework:
\begin{table}[H]
	\begin{center}
		\small \begin{tabular*}{1.1\columnwidth}{p{3cm}p{3cm}p{5.5cm}} 
			\\ \hline \hline
			Role & Who & Role Description \\ \hline \hline

		 	System Designer & The researcher & A researcher designing a mobile context-aware system based in the egocentric interaction paradigm. Uses the framework to design a simulation of the egocentric system.\\ \hline

		 	Simulation User & The researcher, a QA engineer, an end-user of the egocentric system & This role can be fulfilled by users interested in the resulting egocentric system.\\ \hline

		 	Third Party Service & An individual piece of software & An individual piece of software that builds business logic based on the monitored context data, as the simulation unfolds.\\ \hline
		\end{tabular*}
		
		\caption{List of roles for the EgoSim framework}
		\label{table:roles}
	\end{center}
\end{table}
The ''who'' presents the possible user types that can fulfil a specific role, while the ''description'' briefly describes the role's main interest in the system.\\
% section user_roles (end)

%************************************************
\subsection{User Stories}\label{subsec:user_stories}
%************************************************
The following list of user stories, represent the requirements of the EgoSim framework:
\begin{enumerate}
	\item[\textlabel{1.}{us:1}] \emph{As a system designer, I want to model the environment my egocentric system will run in. The environment's model has to be populated with everyday physical objects and devices}. It is part of goal \ref{goal:1} and presented as a need in points \ref{scenario:1}, \ref{scenario:1A}, \ref{scenario:1B} and \ref{scenario:1C}.

	\item[\textlabel{2.}{us:2}] \emph{As a system designer, I need to identify the objects I want to be monitored during the simulation}, as described in goal \ref{goal:2} and point \ref{scenario:1D} in the scenario.

	\item[\textlabel{2.1}{us:2.1}] \emph{As a system designer, I want to specify how certain entities can be interacted with}. Points \ref{scenario:1E}, \ref{scenario:1F}, \ref{scenario:1G} and \ref{scenario:1H} illustrate a few ways entities could be interacted with.

	\item[\textlabel{2.2.}{us:2.2}] \emph{As a system designer, I want to augment objects that will be monitored with information needed by the SSM model}, as illustrated in point \ref{scenario:1I}. In order for the SSM classification to work, it needs to be aware of specific parameters for each entity. By default, each monitored entity will carry a set of default SSM configuration data, but certain entities might need some adjusted values. The SSM configuration parameters are detailed in Section \ref{subsec:ssm_params}.

	\item[\textlabel{3.}{us:3}] \emph{As a system designer, I want to run a simulation based on the environment model I have set up}.

	\item[\textlabel{4.}{us:4}] \emph{As a simulation user, I want to control a virtual avatar within the simulated environment}. This behaviour is mentioned in goal \ref{goal:1} and, as described in points \ref{scenario:2}, \ref{scenario:2A}, \ref{scenario:2B} and \ref{scenario:2C}, I want to be able to control the movement and field of vision of the agent. Also, I want to be able to pick up certain objects, carry them around and make them interact with other objects (e.g. put it down on a surface).

	\item[\textlabel{5.}{us:5}] \emph{As a simulation user, I want to have a sense of reality in the simulated environment}. The user should comprehend and fell present in the target environment.

	\item[\textlabel{6.}{us:6}] \emph{As a system designer, I want the simulation to classify monitored objects into SSM sets as the agent is being controlled}. This requirement is mentioned in goal \ref{goal:2} and highlighted in the scenario throughout points \ref{scenario:2D} and \ref{scenario:2E}. The SSM sets are detailed in Section \ref{subsec:ssm_params}.

	\item[\textlabel{7.}{us:7}] \emph{As a third party service, I want to access the content of SSM sets through a publicly available API from within the simulation}. Goal \ref{goal:3} covers this requirement, while point \ref{scenario:3} describes a useful scenario.

	\item[\textlabel{8.}{us:8}] \emph{As a simulation user, I want to follow the state of SSM sets, in real time, as the simulation unfolds}. As described in goal \ref{goal:3}, this requirement satisfies the need for an ''easy way to visualize the SSM spaces in real-time''.

\end{enumerate}

While an extra requirement is needed to fully support the situative space model - namely, representation of and interaction with virtual devices - we leave that requirement for future work.\\
% section user_stories (end)

%************************************************
\subsection{Non-functional Requirements}\label{subsec:nfrs}
%************************************************
To strengthen the requirements above, we conclude this section with a list of non-functional requirements (NFRs)\footnote{constraints on the system's behaviour}:
\begin{table}[H]
	\begin{center}
		\small \begin{tabular*}{1.1\columnwidth}{p{10cm}p{1.5cm}} 
			\\ \hline \hline
			NFR & Relates to \\ \hline \hline

		 	The agent should not be able to pass through walls and other objects in the environment. & \ref{us:4}\\ \hline

		 	The visualization of simulated environment should be non-blocking. No matter how heavy are the computations carried out under the hood, the simulation user's experience should not be affected. & \ref{us:5}\\ \hline

		 	Only objects within the field of vision of the user should be categorised & \ref{us:6}\\ \hline

		 	The SSM classification should happen in real-time. & \ref{us:6}\\ \hline

		\end{tabular*}
		
		\caption{Non functional requirements}
		\label{table:nfr}
	\end{center}
\end{table}
% section nfrs (end)

\subsection{SSM Sets and Configuration Parameters}\label{subsec:ssm_params}
In this work, we have focused on two main parameters to classify objects around the agent: proximity and field of vision. This means that objects within the agent's field of vision will be classified based on the distance from the agent's current position. We have given a less generic interpretation to the SSM Sets defined in \cite{pederson2011situative}, to fit this use case, as presented in the list below:
\paragraph{World Space} contains the set of all entities (physical objects and mediators) in the environment's virtual model. The framework takes into account only the entities which have been identified to be monitored. Hence, not all visible objects visible are categorized.
\paragraph{Perception Space} is part of space around the agent that is within the agent's field of vision at each moment. Objects within this space that are no further than \emph{PERCEPTION\_DISTANCE} from the agent, will be classified into this set.
\paragraph{Recognition Set} contains the entities that are currently in Perception Space and within their \emph{RECOGNITION\_DISTANCE}. Objects in this space can be directly associated with the agent's activities. For example, a hammer can be perceived as an object up to a certain distance, but when close enough, it can be recognised as a hammer.
\paragraph{Examinable Set} contains the objects that are currently in Perception Space and within their \emph{EXAMINATION\_DISTANCE}. Based on this set it can be determined what actions can be performed with that object. For example, in the perception space a cell phone can be seen as a simple object, in the recognizable set it can be recognized as a mobile device and in the examinable set one can notice it has a screen, a power button and two volume button. We can deduct the type of action one can perform!
\paragraph{Action Space} part of space around the agent that is currently accessible to the agent's physical actions. More specifically, the object has to currently be in Perception Space and within \emph{ACTION\_DISTANCE}. Objects in this set can be directly acted upon.
\paragraph{Selected Set} objects currently being physically handled.
\paragraph{Manipulated Set} objects whose states (external and internal) are currently being changed by the agent. Normally a subset of the Selected Set.\\

As described in the definitions above, in order to correctly classify objects, they need to be augmented with the following parameters: \emph{PERCEPTION\_DISTANCE}, \emph{RECOGNITION\_DISTANCE}, \emph{EXAMINATION\_DISTANCE} and \emph{ACTION\_DISTANCE}. Normally, they should be given meaningful default values, therefore configuring them should not be a mandatory step. Even so, some objects might need adjusted values, so the possibility should be given to the system designer.
% section requirements (end)
%************************************************
\section{Challenges of building onto related systems} % (fold)
\label{sec:reusing_related_systems}
%************************************************
While researching the systems presented in the related work chapter \ref{ch:related_work}, we have analysed the possibilities of reusing some of them so we do not start from scratch. Unfortunately, none of them was fit for our approach. In this section we present the challenges we would face trying to design our solution in the context of some of the related systems.

%************************************************
\subsection{DiaSim}\label{subsec:design_diasim}
%************************************************
After thorough analysis, we found DiaSim's \ref{sec:diasim} simulation model as being generic enough to possibly host our needs. The simulation model seems to be flexible and open for further modifications. The system is not open source and the research team is not yet open for external collaboration. We have contacted the research group at INRIA\footnote{\url{http://www.inria.fr}} proposing a collaboration in order to extend DiaSim in the directions required by this project. The answer we got back is that development of DiaSim is momentarily suspended as most of the group is focused on another project. They will resume development of DiaSim only six month later. Even so, they are yet unable to determine whether a partial release of the sources is possible.\\

Even if building on top of DiaSim would have been possible, some outstanding issues should be tackled. The simulated environment does not monitor the agent's field of vision. To accommodate this need, the renderer module should be entirely replaced. Besides, there is no support for direct interaction with physical objects and mediators. The only type of interaction is proximity based: i.e. a sensor attached to a door would command the door to open if the agent is within range. Moreover, DiaSim is lacking an open API to enable third party service to make use of the simulated system's context data.\\

%************************************************
\subsection{SIMACT}\label{subsec:design_simact}
%************************************************
SIMACT \ref{sec:simact} is a smart home infrastructure simulator built to reproduce everyday life scenarios on a step-by-step basis. To make it easier to comprehend the simulated system, SIMACT animates the simulation using the jMonkey Engine (JME) framework. SIMACT is not a tool to create simulations for simulation user, rather it is oriented towards observing the systems evolution over time.\\

Although SIMACT does not offer the necessary means to support the implementation of our requirements, the JME game engine used to animate the simulations has caught our attention. JME has all the capabilities of a modern game engine which, as discussed in this chapter, we consider well fit to accommodate our requirements. In Chapter \ref{ch:implementation} we will argue for JMonkey as a possible candidate to be used in our work.\\

%************************************************
\subsection{TATUS}\label{subsec:design_tatus}
%************************************************
TATUS \ref{sec:tatus} is a 3D simulator implemented on top of Half-Life's Game Engine. It has exploited the concept \emph{Triggers} from the SDK to simulate sensors and actuators. They are used to generate associated events based on a player's movements and location. Therefore, TATUS already has proximity detection, but only based on sensors.\\

It would be possible to partially base our design on TATUS' triggers, by attaching such a trigger to each object we would like to classify based on the agent's proximity. Moreover, TATUS does not simulate interaction with devices neither does it simulate interaction with everyday physical objects.\\

To conclude, TATUS is not open-source, making it unusable for research outside the institution it was developed in.\\

%************************************************
\subsection{UbiWise}\label{subsec:design_ubiwise}
%************************************************
UbiWise \ref{sec:ubiwise} was designed to simulate the interaction of the agent with the device prototypes within the target environment, not with the actual environment. Moreover, the framework empowers the user to interact with the software running on them. Therefore UbiWise handles the representation of mediators and interaction with virtual objects. This could be a good starting point. But UbiWise is solely oriented towards simulating prototyped devices. The virtual agent can interact with the prototyped devices and with the software running on them, some of the device may even be hand-held. It has no direct interaction with physical objects, the framework does no track the agent position towards objects within the environment and does not determine the objects within the agent's field of vision. The underlying game engine might provide the necessary means to accommodate these requirements, but the effort depends greatly on individual experience in C programming.\\

To conclude, UbiWise was implemented with a radically different goal in mind and it poses an unnecessary technical challenge to try and accommodate EgoSim's requirements into its design.\\

%************************************************
\subsection{Conclusion}\label{subsec:reusing_conclusion}
%************************************************
In this section we have presented the challenges we would have faced trying to design our solution in the context of some of the related systems. None of the system were fit for our approach, but we had much to learn from them, nonetheless. Most systems which allow direct interaction between the agent and the simulated environment were build to some degree using game engines. We are considering to include game engines into our design as argued for in Section \ref{sec:simulation_runtime}.

% section sec:reusing_related_systems (end)
%************************************************
\section{Architecture} % (fold)
\label{sec:architecture}
%************************************************
In this section we will describe the high level components we have developed to accommodate the requirements described in Section \ref{sec:requirements}. To empower the system designer to set up a simulation we need a \emph{Simulation Designer}. Using the simulation designer, the system designer is able to create a \emph{Configured Environment Model}, where all the objects of interest for the simulation have been identified and configured. The \emph{Simulation Runtime} loads a configured environment model and enables the simulation user to control an \emph{Avatar} in order to interact with the environment. As the agent interacts with the environment, the \emph{Context Manager} monitors the agent's position and visual spectrum, delegating the classification of objects around the user to the \emph{SSM Classifier}. The \emph{SSM Sets} can be visualized in the \emph{Context Client} and can be accessed through the \emph{API}.\\

For a better overview, the proposed architecture is illustrated in Figure \ref{fig:initial_architecture}.

\begin{figure}[H]
	\centering
	\includegraphics[width=\linewidth]{gfx/Chapter3/initial_architecture}
	\caption{EgoSim Architecture Diagram}
	\label{fig:initial_architecture}
\end{figure}
% Describe the big picture in words and using a picture as well! Describe the main components imposed by this architecture: 3D modelling (to represent the space), rendering and movement control to build the simulated environment, categorization algorithms AND programming API (including the ContextClient).\\

% In the remaining of this chapter good deep into details about why I've chose certain technologies for each component; put each discussion in its own section! To better argue, present several technologies that could fit the same requirements and argue objectively about the the reasons I've made certain choices. I should refer back to arguments in the related work!

% section architecture (end)
%************************************************
\section{Simulation Designer} % (fold)
\label{sec:simulation_designer}
%************************************************
Creating the environment model is the first step required by our system. As described in requirements \ref{us:1} and \ref{us:2}, the simulation designer needs to empower the researcher to create the model of the target environment and to identify and configure the objects to be monitored.\\

To proceed, first we need to figure out what kind of model we are going to use to represent the environment. Similar projects like DiaSim \ref{sec:diasim} have used a 2D representation, while Tatus \ref{sec:tatus} or UbiWise \ref{sec:ubiwise} have used a 3D representation of the environment. In this context, 2D stands for two-dimensional while 3D stands for three-dimensional.\\

Think from the agent's perspective, a 2D model can cover two dimensions: location and orientation. As mentioned in requirement \ref{us:4}, the simulation user must be able to control the agent's movement (location and orientation) as well as the eyesight (the visual spectrum). The projects which have used a 2D representation were focusing solely on proximity, while the projects aiming for a more realistic interaction have used a 3D model. Such a model can encapsulate more details, proving a better sense of reality as required in \ref{us:5}.\\

With the 3D model in place, the system designer has to identify the objects she/he desires to be monitored and classified during the simulation. To prepare the model as such, we have designed a list of properties each objects will need to carry; some of them are mandatory, while others can have meaningful default values. The properties are detailed in the following list of \textlabel{object properties}{design:object_properties}:
\paragraph{ID} - to uniquely identify an object within the environment. This parameter is mandatory and it is up the system designer to ensure that each object's ID is unique.

\paragraph{Type} - describes the type of the object. We need to support physical objects and devices \ref{us:1}, so this parameter can have two values: Physical or Device.
% section simulation_designer (end)
%************************************************
\section{Simulation Runtime} % (fold)
\label{sec:simulation_runtime}
%************************************************
The aim of this component is to enable the simulation user to interact with the 3D model, created using the simulation designer. As observed in the related work, some of the projects used a 3D model to represent the environment (\ref{sec:simact}, \ref{sec:ubiwise}, \ref{sec:tatus}) and have built the frameworks on various game engines. A game engine represents a software framework designed for the creation and development of video games. Michael Lewis and Jeffrey Jacobson \cite{lewis2002game} argue for the power of game engines: ''The most sophisticated, responsive interactive simulations are now found in the engines built to power games''. Moreover, they argue for the usefulness of game engines in scientific research: ''There are probably as many potential applications for game engines as there are research problems requiring medium-fidelity 3D simulation or high-fidelity interactive graphics. Our hope is to raise awareness of the high-power/lowcost alternative game engines can offer''.\\

Our design proposal for the simulation runtime is to use a game engine as a starting point. There is wide range of available game engines written in different programming languages, some open-source, others proprietary, most of them providing out of the box components to be reused for specific programming tasks. In conclusion, game engines can fit the various needs and programming experience of the research group designing a simulation for research. The overall design solution we provide based on a game engine is not tied to a certain implementation, but it uses general concepts that can be found in most modern game engines.\\

To conclude this section, we will briefly discuss the architecture of a game engine and present the most important concepts used in game engines which are relevant for this work.\\

%************************************************
\subsection{Architecture of a Game Engine}\label{subsec:game_engine_architecture}
%************************************************
Analysing the game engine's structure depicted in Figure \ref{fig:game_engine_structure}, we can deduct that the game engine renders the virtual environment represented as a 3D or 2D model, while the game code is the custom behaviour for a specific game. Moreover, support for client-server behaviour (usually for multi-player games) can be implemented through the network code module. The fact that the game engine runs at a high-level within an operating system, above the system drivers, makes the client modules platform-independent. The game engine works just like an interpreter, running the same client code (game) on all the supported platforms. Hence, the game engine comes as a solution for requirement \ref{us:3}.\\
\begin{figure}[H]
	\centering
	\includegraphics[width=\linewidth]{gfx/Chapter3/game_engine_structure}
	\caption{Modular game engine structure \cite{lewis2002game}}
	\label{fig:game_engine_structure}
\end{figure}

''A game engine includes all elements in Figure \ref{fig:game_engine_architecture} that have no effect on actual content, that is, everything indicated by dashed lines plus an event loop'' \cite{shantz1998designing}. Therefore, the core features provided by a game engine include a graphics module (rendering engine for 3D graphics), input control (communication with peripherals like mouse, keyboard, joystick, etc), dynamics (e.g. approximate simulation of certain physical systems, such as rigid body dynamics, fluid dynamics, etc), sound. The \emph{event handler}, \emph{game logic} and \emph{level data} (environment model) are client modules that are used to implement a certain type of game / simulation.\\
\begin{figure}[H]
	\centering
	\includegraphics[width=\linewidth]{gfx/Chapter3/game_engine_architecture}
	\caption{Game engine architecture \cite{shantz1998designing}. The components marked with dashed lines, are contained within the game engine, while the one marked with solid lines (except for the platform) represent client code.}
	\label{fig:game_engine_architecture}
\end{figure}
% subsection architecture_of_a_game_engine (end)

%************************************************
\subsection{Game Engine Concepts}\label{subsec:game_engine_concepts}
%************************************************
The graphics module (renderer) interprets the environment model as a \emph{\textlabel{scene graph}{scene_graph}}, depicted in Figure \ref{fig:scene_graph}. The scene graph is a tree-structured graph, representing the programmatic model of the environment inferred from the simulated environment's virtual model created with the Simulation Designer \ref{sec:simulation_designer}. Besides the information required for rendering, some nodes should carry the Ego metadata the system designer might have attached in the design process. The nodes in the scene graph are the entities used in programming tasks carried out within the game logic.
\begin{figure}[H]
	\centering
	\includegraphics[width=\linewidth]{gfx/Chapter3/scene_graph}
	\caption{Scene graph \cite{shantz1998designing}}
	\label{fig:scene_graph}
\end{figure}

''Missing from Figure \ref{fig:scene_graph} is the \emph{\textlabel{camera}{camera}}, an object not part of the scene graph [...]. The root of a scene graph is attached to a camera when both are initialized''\cite{shantz1998designing}. Using the information of a certain camera (position and orientation) and the scene graph, the renderer computes how to draw the 3D scene graph to the 2D screen. This is what the user sees on the computer screen as the game unfolds. We can think of a camera positioned at a certain point and having a certain direction as the user's field of vision, displaying what the user would actually see if she/he would be located as such. In a game, there can be multiple cameras, providing different perspectives.\\

The mathematical computations used by the graphics engine are all based on absolute or relative coordinates of the entities. Coordinates represent a location in a coordinate system. 3D game engines use a 3D coordinate system. As illustrated in Figure \ref{fig:3d_coordinates_system}, coordinates are relative to the origin at (0, 0, 0). In 3D space, we need to specify three coordinate values to locate a point: X (right), Y (up), Z (towards us). Similarly, -X (left), -Y (down), -Z (away from us).\\
\begin{figure}[H]
	\centering
	\includegraphics[width=\linewidth]{gfx/Chapter3/coordinate_system}
	\caption{A 3D coordinate system \cite{jme3_terminology:online}}
	\label{fig:3d_coordinates_system}
\end{figure}

As mentioned above, the origin is the central point in the 3D world, where the three axes meet, at the coordinates (0, 0, 0). A coordinate represents a location within the given coordinate system. To represent a direction, game engines use the concept of vectors. A \emph{\textlabel{vector}{vector}} has a length and a direction, just like an arrow in 3D space. A vector starts at a coordinate (x1, y1, z1) or at the origin, and ends at the target coordinate (x2, y2, z2). Backwards directions are expressed with negative values.\\

With this concrete information, let's take a step back to reflect on the concept of a camera. As we have already mentioned, the camera is made up by a location and a direction. The region of space that appears on the computer screen at a certain point is called the \emph{\textlabel{view frustum}{view_frustum}}. This is basically the \emph{field of vision} of the active camera and it is illustrated in Figure \ref{fig:view_frustum}.\\
\begin{figure}[H]
	\centering
	\includegraphics[width=\linewidth]{gfx/Chapter3/view_frustum}
	\caption{View Frustum or Field of Vision (\url{http://blog.jaxgl.com/2012/06/render-smarter})}
	\label{fig:view_frustum}
\end{figure}

So, the camera contains both information about the location and direction of the camera within the 3D world, as well as information about the view frustum of the camera. The game engine uses the camera and the scene graph to determine which objects to draw and which not to. This concept is called \emph{culling}.\\

Another useful term in game engines is the \emph{\textlabel{bounding volume}{bounding_volume}}. 3D objects can have irregular forms, other that basic 3D geometrical shapes. To aid mathematical computation, they are wrapped in a bounding volume which pragmatically represents the object as a sphere, cube, cylinder, etc. The bounding volume is then used to compute physical interactions (where physical objects collide, push and bump off one another), and non-physical collisions (mathematical intersections).\\

Finally, the concept of \emph{\textlabel{ray casting}{ray_casting}} is useful to compute distances. Imagine the ray casts a line starting in a certain point of the 3D space having either a finite or infinite length. This ray can be used to determine what objects in the 3D space it intersected. With this information we can determine distance to certain objects.\\
% subsection game_engine_concepts (end)

%************************************************
\subsection{Simulation Runtime Summary}\label{subsec:simulation_runtime_summary}
%************************************************
To summarise, based on the research presented in this section we decided to base the simulation runtime on a game engine. In the last part of this section we have presented the most relevant concepts for our work in the game engines terminology. The \ref{scene_graph} represents a programmatic data structure of the simulated 3D world. The field of vision within the 3D space is provided by the \ref{camera} based on the \ref{view_frustum}, which is computed using the current \ref{vector} (location and direction) of the \ref{camera}. To aid mathematical computations, the 3D representations of objects are wrapped in a bounding volume, regulating their form to basic 3D geometrical shapes. Last, to compute distances the concept of \ref{ray_casting} comes in handy.\\

The design of the following components will use the game engine components and terminology presented in this section.
% subsection simulation_runtime_summary (end)

% \subsection{The Monitoring Service}\label{subsec:monitoring_service}
% \subsection{The API}\label{subsec:api}
% \subsection{The ContextClient}\label{subsec:context_client}
% section simulation_runtime (end)
%************************************************
\section{The Agent} % (fold)
\label{sec:the_agent}
%************************************************
The agent is the active entity the simulation user can control, using various input mechanisms, in order to interact with the environment. In 3D games and simulations, there are two main graphical perspectives employed to interact with the environment -- the first-person and third-person perspectives. These perspectives are the result of positioning the camera in a certain manner, providing a different perception of the environment. To control the agent's position and orientation, the camera's position and direction are coupled to a set of input controls. As the camera is moved and rotated, it creates the impression of interaction with the 3D environment. For a more realistic interaction, various animations are carried out.

\subsection{First-Person}\label{subsec:first_person}
The first-person view is a graphical perspective rendered from the viewpoint of the agent. Figure \ref{fig:fps} illustrates the first-person view from a running simulation built with Tatus \cite{o2005testbed}.

\begin{figure}[H]
	\centering
	\includegraphics[width=\linewidth]{gfx/Chapter3/fps}
	\caption{Example of a first-person perspective}
	\label{fig:fps}
\end{figure}

Using this perspective, the simulation displays what the simulation user would see with the agent's own eyes thus, objects have to be scaled accurately to appropriate sizes. In this perspective, the simulation users typically cannot see the agent's body, although it might be helpful for the user to see the agent's hands. On the plus side, there is no need for sophisticated animations for the agent, which reduces development time. Also, it helps for easier, more realistic aiming (pointing at objects to interact with) since there is no representation of the avatar to block the simulation user's view. Not seeing the agent's body reduces development efforts, but might represents a drawback as well.\\

\subsection{Third-Person}\label{subsec:third_person}
The third-person view is a graphical perspective rendered from a fixed distance behind and slightly above the virtual agent. The third-person perspective can provide an animated, strong characterized agent, directing the simulation agent's attention as she/he were watching a film. Figure \ref{fig:req_third_person} illustrates and example of a third-person from the game Second Life\footnote{\url{http://secondlife.com}}.
\begin{figure}[H]
	\centering
	\includegraphics[width=\linewidth]{gfx/Chapter3/third_person}
	\caption{Example of a third-person perspective}
	\label{fig:req_third_person}
\end{figure}

On the plus side, the third-person perspective allows the simulation user to see the area surrounding the agent more clearly. This viewpoint facilitates more interaction between the virtual agent and their surrounding environment.\\

As a drawback, the third-person perspective can interfere with accurately pointing at objects, as the agent's body might block the simulation user's view. Moreover, implementing this perspective takes a lot more effort as there is a need for an animated agent body and for custom code to make the camera follow the agent within the simulation.\\

\subsection{Discussion}\label{subsec:agent_discussion}
Based on the preceding review of existing perspectives, we have decided to include in our design a first-person perspective for the agent. An argument for choosing this perspective is that a first-person view provides the simulation user with greater immersion into the simulated environment.\\

Avatar based games and simulations usually provide both views, with an easy way to switch perspectives during runtime. For this project, a third-person perspective could be useful for future work if we are to represent wearable devices. Also, it enables the simulation user to observe the various body gestures the agent is doing.\\

To fully support requirement \ref{us:4}, we need to define a way of controlling the agent. Most games and simulations use a combination of the keyboard and the mouse to control the agent. The standard control combination is described bellow:
\begin{itemize}
	\item moving \emph{the mouse} controls the agent's direction; hence mouse movements translate into rotating the agent's viewpoint within the environment. This control is used to look around and to point at objects the simulation user desires to interact with
	\item clicking \emph{the left mouse button} triggers an interaction. The interaction is contextual; for example, clicking on a switch will turn on or off the switch. If the user clicks on a pen, the agent would pick the pen up, while if the user clicks on a surface while the agent is carrying a pen, the agent will put the pen onto the surface. Of course, the ability to carry out these actions depends on the distance between the object and the agent
	\item moving the agent is done by activating a set of four keys on the keyboard. We have chosen the standard W, A, S, D key combination. Each key moves the agent towards a certain direction, Therefore, W - forward, A - left, D - right, S - backward.
\end{itemize}

To help the simulation agent with the task of accurately aiming at objects, a cross icon will always be present in the middle of the screen. The object currently pointed at will represent the object the user intends to interact with.\\

In order to determine the object to interact with, we use the concept of \ref{ray_casting}. To achieve this, we cast a ray of infinite length from the current location of the camera along it's direction. This ray passes right through the middle of the cross and intersects all the objects along the way. We then determine the object to interact with by picking the closest object to the agent; that is the first object intersected by the ray. Moreover, the object can be interacted with only if it carries Ego metadata, otherwise the interaction is not possible. We have imposed this constraint because without the metadata we cannot take concrete decision on what to do with the object.\\

Whether an object can be interacted with or not, is given by the current distance towards that object. Therefore, the distance has to be at most equal to the value of the ACTION\_DISTANCE property in the metadata.\\

Finally, the agent should not be able to pass through walls and other physical objects. Game engines provide various mechanisms for collision detection that can be used in the implementation to accommodate this aspect.\\
% section the_agent (end)
%************************************************
\section{Monitoring Service} % (fold)
\label{sec:monitoring_service}
%************************************************

1. keeping track of the agent's position\\
2. determining visible objects\\
3. monitoring distance to each object -- ray casting, custom controls\\

4. all computations based on the Ego Meta-data

% section monitoring_service (end)
%************************************************
\section{The API} % (fold)
\label{sec:api}
%************************************************
In this section we will present the design for an API, accessible by third party services. As discussed in requirement \ref{us:7}, the API should provides real-time access to the SSM sets. The nature of third party service is that it is entirely decouple from the application providing the API, it is highly probable to be written in another programming language and could be running on any another software platform. To accommodate all these aspects we need to provide a loosely coupled, highly interoperable API, one that does not contain the client service to certain programming languages or platforms.\\

An example of a system that successfully achieved a high degree of interoperability, through a fixed interface, is the Web \cite{coulouris2005distributed}. In \cite{kindberg2001web} the authors manage to point out the fast growth of the Web towards an ubiquitous environment. They propose an architectural style for this distributed, volatile environment: the REpresentational State Transfer (REST). It is a set of principles that, when correctly applied, help building software architectures and applications that benefit from all the qualities of the Web. Those qualities include greater scalability, efficient network usage and independent evolution of clients and servers -- also called loose coupling.\\

This led us to adopt the REST architectural style and integrate it in EgoSim's interoperation mechanism. The entities used to communicate over REST are represented in the JavaScript Object Notation (JSON) encoding. This design choice ensures loose coupling and a reliable communication between components, making EgoSim easily extensible and opened for communication even with services written in other programming languages.\\

The JSON format for data transfer is a lightweight data-interchange format, it is easy for humans to read and write and it is easy for machines to parse and generate. We have decided to use JSON over other existing data-interchange formats as it is the one of the most lightweight type of structured data. For example, the Extensible Markup Language (XML) is another type of data-interchange format, but it is often criticised for its verbosity and complexity.\\

Therefore, the content of the sets will be available through the RESTful API in the JSON format. As the SSM sets are made up by the monitored physical objects and mediators, the sets will contain the JSON representation of the Ego metadata (including the LAST\_MEASURED\_DISTANCE) property.\\
%************************************************
\subsection{RESTful API} % (fold)
\label{sec:rest_api}
%************************************************
REST is an architecture style for distributed hypermedia systems. A web service is an API which is accessed through the HyperText Transfer Protocol (HTTP) and executed on a remote system, hosting the requested service. A RESTful web service is a web service implemented using HTTP and the principles of REST. The RESTful web service is defined by a collection of resource, each of which is defined by three main characteristics:
\begin{itemize}
	\item the base URI identifying the web service
	\item the Multipurpose Internet Mail Extensions (MIME) type of the data supported by the web service (JSON, XML, etc)
	\item the web service's interface defined against the HTTP supported methods like POST, GET, PUT, DELETE etc
\end{itemize}

The REST architectural style imposes a client-server architecture, which fits well the requirement for a decoupled design we are aiming for.\\
% subsection rest_api (end)

%************************************************
\subsection{Discussion} % (fold)
\label{sec:api_discussion}
%************************************************
One of the drawbacks of the RESTful approach is that the third party service has to access every time it needs information. So, if a service would like to take an action in real-time, it would need to access the RESTful API quite often. Even so, there will always be a \emph{latency problem} -- the client service will not receive a change in the SSM sets the instance it occurs. We have found this drawback acceptable for the current work. A service could be able to access the REST service every second which might provide a close to real-time access to contextual changes. At the moment we want to explore how third party service would use the data provided by the API and how they would build business logic around it. Based on this information, in future work we can improve the API to provide client services with real-time access to changes in the SSM sets.\\ 
% subsection api_discussion (end)

% section api (end)
%************************************************
\section{The Context Client} % (fold)
\label{sec:context_client}
%************************************************

1. a third party app using the api\\
2. a web page served from within the simulation --> polling vs. web sockets\\

% section context_client (end)
%************************************************
\section{The Design} % (fold)
\label{sec:the_design}
%************************************************

% section the_design (end)

% \begin{enumerate}
% 	\item Follow a modular design to implement the core of the framework as a set of decoupled components:
% 		\begin{itemize}
% 			\item The \emph{Data Model}. This represents the contextual entities and the relationship between them that the simulator will monitor, briefly detailed in Section \ref{sec:data_model}.
% 			\item The \emph{World Model}. Represents the current status of the simulated environment (e.g. position of the agent, status of a certain switch etc.). The structure of the entities in this model have been defined in the Data Model.
% 			\item A universal communication protocol to access the core. It needs to be universal so that other components, can easily be developed and integrated.
% 			\item Query and modify API of information in the World Model. To make the integration between the core of the framework and the other modules, these APIs will need to provide a complete set of methods.
% 		\end{itemize}
% 	\item Evaluate the framework:
% 		\begin{itemize}
% 			\item Implement a module to modify the state of the simulated environment (DummySim). This comprises of manipulating the agent's parameters and performing actions like picking up an item, touching a screen etc. The purpose of this module is to evaluate the modify API of the core. It also servers as a tool to modify the environment which is a required input for the SSM model to be evaluated.
% 			\item Implement the \emph{SSM Model}. Using the current World Model's query API, retrieve the current set of contextual data and apply the rules defined in the SSM. This component will actively monitor the current status of the simulator, having as a result an up to date collection of SSM entity sets: world space, perception space, recognizable set, examinable set, action space, selected set and manipulated set. This evaluation will be considered successful if the results are similar to the results obtained in the initial evaluation of the SSM described in \cite{pederson2011situative}.
% 		\end{itemize}
% \end{enumerate}


% In order to allow the user of the simulator to easily interact with the simulated environment, the visualization is really important. Projects that simulate real world scenarios, employ game alike solution providing complex graphics and interaction between the user of the simulator and the simulated world.\\

% This work represents initial research for this egocentric simulator. Therefore, my main focus is on the framework's core. But, I will invest time in researching existing solutions, either 2D or 3D; if I will document the suitable frameworks with high potential. This is nice-to-have. I don't want to set it as a goal of this project, rather talk about it as future work.

% \subsection{Data Model}\label{sec:data_model}
% The data model is aimed at defining the supported entities and the relationships between them. In this work the aim is to support physical objects, virtual objects and mediators.\\

% The agent represents a virtual character that can freely moved around in the environment. Beside the position the of the agent, the user can also control more complex parameters of the agent like orientation (affects field of vision), sitting down / standing, hands orientation etc. The agent will be also be able to carry out a certain set of actions like picking objects up, putting objects down etc.\\

% Physical objects will be described as entities with a certain set of properties (location, weight etc.). Entity composition will be supported to some extend: place objects on top of each other (i.e. a stapler on a table), a mobile phone in the backpack worn by the agent.\\

% This section created a brief overview of the supported entity types and relationships. They will be described in details in Chapter \ref{ch:design}.