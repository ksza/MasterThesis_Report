%************************************************
\section{The API} % (fold)
\label{sec:api}
%************************************************
In this section we will present the design for an API, accessible by third party services. As discussed in requirement \ref{us:7}, the API should provides real-time access to the SSM sets. The nature of third party service is that it is entirely decouple from the application providing the API, it is highly probable to be written in another programming language and could be running on any another software platform. To accommodate all these aspects we need to provide a loosely coupled, highly interoperable API, one that does not contain the client service to certain programming languages or platforms.\\

An example of a system that successfully achieved a high degree of interoperability, through a fixed interface, is the Web \cite{coulouris2005distributed}. In \cite{kindberg2001web} the authors manage to point out the fast growth of the Web towards an ubiquitous environment. They propose an architectural style for this distributed, volatile environment: the REpresentational State Transfer (REST). It is a set of principles that, when correctly applied, help building software architectures and applications that benefit from all the qualities of the Web. Those qualities include greater scalability, efficient network usage and independent evolution of clients and servers -- also called loose coupling.\\

This led us to adopt the REST architectural style and integrate it in EgoSim's interoperation mechanism. The entities used to communicate over REST are represented in the JavaScript Object Notation (JSON) encoding. This design choice ensures loose coupling and a reliable communication between components, making EgoSim easily extensible and opened for communication even with services written in other programming languages.\\

The JSON format for data transfer is a lightweight data-interchange format, it is easy for humans to read and write and it is easy for machines to parse and generate. We have decided to use JSON over other existing data-interchange formats as it is the one of the most lightweight type of structured data. For example, the Extensible Markup Language (XML) is another type of data-interchange format, but it is often criticised for its verbosity and complexity.\\

Therefore, the content of the sets will be available through the RESTful API in the JSON format. As the SSM sets are made up by the monitored physical objects and mediators, the sets will contain the JSON representation of the Ego metadata (including the LAST\_MEASURED\_DISTANCE) property.\\
%************************************************
\subsection{RESTful API} % (fold)
\label{sec:rest_api}
%************************************************
REST is an architecture style for distributed hypermedia systems. A web service is an API which is accessed through the HyperText Transfer Protocol (HTTP) and executed on a remote system, hosting the requested service. A RESTful web service is a web service implemented using HTTP and the principles of REST. The RESTful web service is defined by a collection of resource, each of which is defined by three main characteristics:
\begin{itemize}
	\item the base URI identifying the web service
	\item the Multipurpose Internet Mail Extensions (MIME) type of the data supported by the web service (JSON, XML, etc)
	\item the web service's interface defined against the HTTP supported methods like POST, GET, PUT, DELETE etc
\end{itemize}

The REST architectural style imposes a client-server architecture, which fits well the requirement for a decoupled design we are aiming for.\\
% subsection rest_api (end)

%************************************************
\subsection{Discussion} % (fold)
\label{sec:api_discussion}
%************************************************
One of the drawbacks of the RESTful approach is that the third party service has to access every time it needs information. So, if a service would like to take an action in real-time, it would need to access the RESTful API quite often. Even so, there will always be a \emph{latency problem} -- the client service will not receive a change in the SSM sets the instance it occurs. We have found this drawback acceptable for the current work. A service could be able to access the REST service every second which might provide a close to real-time access to contextual changes. At the moment we want to explore how third party service would use the data provided by the API and how they would build business logic around it. Based on this information, in future work we can improve the API to provide client services with real-time access to changes in the SSM sets.\\ 
% subsection api_discussion (end)

% section api (end)